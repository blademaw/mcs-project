\section{Introduction}\label{sec:introduction}

%Vector-borne diseases (VBDs) such as malaria, dengue, and chikungunya are major public health concerns, representing approximately 17\% of all infectious diseases and causing over 700,000 deaths annually \cite{world_health_organisation_who_vector-borne_2020}.  While vector control methods such as mosquito repellents and insecticide-treated bed nets effectively reduce disease transmission, these methods rely primarily on self-participation and voluntary adoption, which are driven by individuals' perceived risk of infection and personal behavioural attitudes. Mathematical models have become increasingly popular to help health practitioners anticipate downward pressure on health infrastructure and guide policy decisions for outbreak responses \cite{reiner_systematic_2013}. However, these models assume homogeneity of populations and thus cannot incorporate heterogeneous human behavioural attitudes that influence participation in vector control. Individual- or agent-based modelling is a computational technique that simulates the individual components (\textit{agents}) of a system to generate complex phenomena indirectly \cite{epstein_growing_1996}, offering a solution for directly encoding human behaviour. Despite the existing success of agent-based models in VBD applications, however, there is a lack of research investigating the dynamics between VBD spread and behaviours towards the use of preventive vector control tools.

Vector-borne diseases (VBDs) such as malaria, dengue, and chikungunya are major public health concerns, representing approximately 17\% of all infectious diseases worldwide \cite{world_health_organisation_who_vector-borne_2020}. While preventive measures such as mosquito repellent and insecticide-treated bed nets can effectively reduce disease transmission, these methods rely primarily on self-participation, which is driven by individuals' perceived risk of infection and personal behavioural attitudes. Mathematical models have become increasingly popular to anticipate downward pressure on health infrastructure and guide policy decisions for outbreak responses \cite{reiner_systematic_2013}. However, these models assume homogeneous populations, and thus cannot incorporate heterogeneous human behavioural attitudes that influence preventive measure adoption. Agent-based models (ABMs) are computational tools that simulate the individual components (\textit{agents}) of a system to generate complex phenomena indirectly \cite{epstein_growing_1996}, offering a solution for directly encoding human behaviour. Despite the existing success of ABMs in VBD applications, however, most of these models neglect this component and instead opt for simple models of human behaviour. While there is a wealth of literature from psychology and sociology to inform mechanisms of behaviour change in ABMs, there remains a lack of research that investigates how computationally encoding conceptual theories of behaviour in such models can capture complex patterns of preventive measure adoption and VBD spread.%Overall, there is a lack of research that investigates whether drawing on theories of behaviour from psychology and sociology can inform mechanisms of decision-making in ABMs of VBDs, capture complex patterns of preventive measure adoption and disease spread.

With more than 80\% of the global population living in areas vulnerable to at least one VBD \cite{golding_integrating_2015}, these diseases are major threats to worldwide public health infrastructure. Not only do VBDs result in epidemics which place large burdens on health institutions, but they can also stunt the economic development of communities, disproportionately affecting poorer nations \cite{lum_cost_2009, degroote_interventions_2018}. To curb the spread of VBDs, health officials employ \textit{vector control} to limit disease transmission from arthropods (such as mosquitoes, sandflies, and ticks) to humans. These chemical and non-chemical interventions such as mosquito repellent, insecticide-treated bed nets, and long-sleeved clothing are the primary---and often \textit{only}---strategy for reducing VBD spread due to a lack of a vaccine or preventive drug \cite{wilson_importance_2020}.

Although modern-day vector control methods can effectively reduce disease transmission, they primarily rely on self-participation and voluntary adoption of preventive measures, which are influenced by individuals' perceived risk of infection and personal behavioural attitudes \cite{winch_effectiveness_1992, brewer_risk_2004, raude_public_2012, lopes-rafegas_contribution_2023}. Risk perception and \textit{preventive behaviours} (behaviours that involve the use of preventive measures) are often recognised as profoundly important for understanding VBD spread, yet remain an understudied area \cite{williams_role_2010}. With additional issues surrounding vector control such as rising mosquito resistance to insecticides \cite{chala_emerging_2021, hemingway_averting_2016}, labour and funding disputes \cite{winch_effectiveness_1992}, and insufficient coverage and access within communities \cite{okumu_what_2022}, understanding how to motivate at-risk populations to adopt vector control tools for protection is an important public health concern. \\

To help health practitioners address VBD spread, mathematical models have become increasingly popular to guide policy and interventions by modelling the interactions between vector and human populations. These are equation-based models (EBMs), which express the dynamics of systems holistically through relationships in system variables \cite{van_dyke_parunak_agent-based_1998}, and provide utility in their tractable specification and analytical nature. In such models, however, risk perception and preventive behaviours have historically been neglected, despite their significance. This is largely due to the assumption of EBMs that populations are homogeneous or \textit{well mixed}, meaning heterogeneous factors of systems cannot be readily incorporated \cite{perkins_heterogeneity_2013, hunter_comparison_2018}.

Alternative mathematical methods for modelling VBDs also exist, such as statistical or machine learning approaches. For example, \citet{aerts_understanding_2020} used structural equation modelling to estimate the association between individuals' risk perception of VBDs and their tendencies to protect themselves. \citet{pley_digital_2021} highlighted artificial intelligence techniques as promising tools for VBD surveillance systems due to their ability to ingest data quickly and accurately predict epidemic outbreaks. However, such approaches often focus solely on prediction and require large amounts of data to be feasible. In VBD contexts, this is not always the case, as the value of modelling often instead lies in the ability to explain why certain dynamics arise, to illuminate core interactions of at-risk populations, and to discover new questions about behaviours that evolve alongside disease spread \cite{epstein_why_2008, bruch_agent-based_2015}. Ultimately, VBD models require an appropriate mechanistic underpinning to sufficiently consider complex social, community, and contextual human actions \cite{funk_modelling_2010, bedson_review_2021}.

Agent-based modelling is a \textit{bottom-up} computational modelling technique that simulates the individual components (\textit{agents}) of a system to reproduce emergent behaviour through stochastic interactions \cite{epstein_growing_1996}. Simulating systems at the individual level means heterogeneous factors of agents---such as risk perception and behavioural attitudes---can be naturally incorporated within models. Additionally, due to their modular implementation, agent-based models (ABMs) allow practitioners to represent hypothetical interventions without overhauling model architecture, unlike EBMs \cite{axtell_agent-based_2022}. This capacity for \textit{in silico} experimentation is especially useful in the context of VBDs, where analysing hypothetical interventions or counterfactual scenarios can be difficult given the data available, or lack thereof. The flexible nature of ABMs has spawned a growing body of research using the approach to model VBD spread \cite{krzhizhanovskaya_agent-based_2020, selvaraj_vector_2020, manore_network-patch_2015, linard_multi-agent_2009, jacintho_agent-based_2010, perkins_agent-based_2019, mulyani_agent_2017, maneerat_spatial_2016} and study the effects of risk perception and preventive behaviours on various infectious diseases \cite{mao_modeling_2014, kandiah_empirical_2017, du_how_2021, tully_coevolution_2013, andrews_disease_2015}. \\

There is sparse research, however, that investigates how realistic representations of human behaviour can be integrated within ABMs to capture complex patterns of preventive behaviours and disease spread. While simple representations of behaviour have previously been implemented in ABMs to study the dynamics of self-protection alongside epidemics \cite{mao_modeling_2014, barbrook-johnson_uses_2017, du_how_2021, scheidegger_agent-based_2017, mateus_c_modeling_2021}, these models have historically used simple threshold models of behaviour inspired by rational choice theory \cite{bedson_review_2021, janssen_empirically_2006, verelst_behavioural_2016}. Theoretical behaviour change theories from psychology and sociology are rarely used to inform decision-making processes in ABMs \cite{weston_infection_2018}, and little attention has been paid to the differences between computational implementations of these theories.

This research gap is recognised in the field as a missed opportunity, since drawing upon the wealth of theoretical behaviour change literature can arguably incorporate more nuanced representations of decision-making \cite{weston_infection_2018, funk_modelling_2010}. A deeper understanding of how preventive behaviours evolve within at-risk individuals may also aid the design of health interventions, and thus warrants further research into the implementation and impacts of conceptual behaviour change theories in ABMs.

%Ultimately, if modellers wish to embed realistic decision-making processes into infectious disease models, further research into the implementation and effects of conceptual behaviour change theories is needed.

%If policymakers and health officials wish to understand the interactions between disease spread and preventive behaviours, further research into epidemiological models that embed psychologically grounded behavioural processes \hlnote{is necessary}{Potentially too strong---reword to \q{has merit} or \q{is valuable}?}.

%There is sparse research, however, that employs ABMs to investigate the psychological mechanisms of behaviour change toward preventive measures for infectious diseases. While \hlnote{theories of behaviour change}{Terminology is not quite right (\q{behavioural change theory} is proper), but does this matter for introduction?} from psychology such as the Health Belief Model \cite{becker_health_1974} and Protection Motivation Theory \cite{rogers_protection_1975} have previously been implemented in ABMs \cite{abdulkareem_intelligent_2018, abdulkareem_risk_2020, durham_incorporating_2012, karimi_effect_2015}, few research efforts have attempted to quantify the impacts of employing different behaviour change theories within epidemiological models, despite existing literature suggesting their importance \cite{scheidegger_agent-based_2017, mateus_c_modeling_2021}. If policymakers and health officials wish to understand the interactions between disease spread and preventive behaviours, further research into epidemiological models that embed psychologically grounded behavioural processes \hlnote{is necessary}{Potentially too strong---reword to \q{has merit} or \q{is valuable}?}.

%how CBIs affect the spread of VBDs alongside the adoption of preventive measures, despite the fact that active community engagement has been shown to be necessary for effective vector control campaigns \cite{winch_effectiveness_1992, rivera_adoption_2023}. Furthermore, while psychological behavioural theories of decision-making such as the Health Belief Model \cite{becker_health_1974} and Protection Motivation Theory \cite{rogers_protection_1975} have previously been implemented in ABMs to simulate preventive behaviours \cite{abdulkareem_intelligent_2018, abdulkareem_risk_2020}, the area remains understudied, especially in the context of VBDs. Lastly, few research efforts have investigated the impacts of using different behavioural theories on disease spread in models, despite existing literature suggesting the importance of incorporating decision-making processes in ABMs \cite{scheidegger_agent-based_2017, mateus_c_modeling_2021}.

%Although the field is still evolving, ABMs have become popular tools for infectious disease modelling in public health contexts where heterogeneity is important \cite{tracy_agent-based_2018}. There is a growing body of research around VBD modelling for various geographical and spatial regions \cite{krzhizhanovskaya_agent-based_2020, selvaraj_vector_2020, manore_network-patch_2015, linard_multi-agent_2009, jacintho_agent-based_2010, perkins_agent-based_2019, mulyani_agent_2017, maneerat_spatial_2016} and the effects of risk perception and preventive behaviours on other infectious diseases \cite{mao_modeling_2014, kandiah_empirical_2017, du_how_2021, tully_coevolution_2013, andrews_disease_2015}. There is sparse research, however, that investigates how CBIs affect the spread of VBDs alongside the adoption of preventive measures, despite the fact that active community engagement has been shown to be necessary for effective vector control campaigns \cite{winch_effectiveness_1992, rivera_adoption_2023}. Furthermore, while psychological behavioural theories of decision-making such as the Health Belief Model \cite{becker_health_1974} and Protection Motivation Theory \cite{rogers_protection_1975} have previously been implemented in ABMs to simulate preventive behaviours \cite{abdulkareem_intelligent_2018, abdulkareem_risk_2020}, the area remains understudied, especially in the context of VBDs. Lastly, few research efforts have investigated the impacts of using different behavioural theories on disease spread in models, despite existing literature suggesting the importance of incorporating decision-making processes in ABMs \cite{scheidegger_agent-based_2017, mateus_c_modeling_2021}.

%The over-reliance on vector control commodities combined with the historical failings of global health institutions to effectively respond to emerging diseases in a timely manner \cite{bardosh_addressing_2017} has, in recent years, motivated a community-oriented approach to VBD spread. Community-based interventions (CBIs) are targeted approaches to address the expansion and emergence of VBDs by emphasising the \q{needs and capacities of vulnerable population groups and local stakeholders} \cite{bardosh_addressing_2017}. Examples include community empowerment campaigns, ecosystem management projects, and waste management programs \cite{perez_realist_2021}.

%While CBIs can ultimately increase vector control efficacy and reduce VBD severity, they primarily rely on influencing community engagement and participation in preventive measures, which are influenced by individuals' behavioural attitudes and perceived risk of infection \cite{winch_effectiveness_1992, brewer_risk_2004, raude_public_2012, lopes-rafegas_contribution_2023}. Therefore, in order to effectively support self-protection against VBDs within at-risk communities, policymakers need to understand the interaction between disease spread, risk perception, and \textit{preventive behaviours}---behaviours that involve the use of preventive measures.


\clearpage
\subsection{Research aims, question, and objectives}\label{sec:research-aims}

The research gap in the literature outlined above---the understudied effects of extending representations of human behaviour in ABMs for VBDs---motivates the research aims of this project. Specifically, this project aims to use computational models to quantify and investigate how integrating behaviour change theories within ABMs influences the adoption of preventive measures, and consequently, VBD spread. To make progress towards these aims, I investigate the following research question:

\vspace{.5cm}
\noindent\textit{How does representing human behaviour in computational models of vector-borne disease spread impact simulated infection dynamics and the estimated efficacy of health interventions?}
\vspace{.5cm}

In particular, I investigate how incorporating complex dynamics of preventive behaviours into a general model of VBD spread influences the adoption of preventive measures, such as insecticide-treated bed nets. Additionally, I compare how different behaviour change theories can be computationally represented in such models, and what their similarities and differences are when used to simulate preventive behaviours. In addressing this research question, I contribute a methodological study to the field of VBD modelling and analyse the impacts of different behavioural theories that describe psychological processes of protection motivation.

Specifically, the research objectives I set out to achieve within this thesis are:

\begin{objectives}[itemindent=4em]

\item Extend an established ABM of a VBD outbreak with preventive measures and contextual information from survey data.\label{ro1}

\item Parameterise the ABM with empirical data to contextualise the model to a real-world scenario.\label{ro2}

\item Computationally encode two behaviour change theories as agent decision-making processes.\label{ro3}

\item Analyse and compare the impacts of different behaviour change theories on preventive behaviours and disease spread for VBDs.\label{ro4}

\item Simulate hypothetical scenarios for each behaviour change theory to quantify the differences in output of disease spread and protection dynamics.\label{ro5}

\end{objectives}

\clearpage
\subsection{Thesis structure}

The structure of this thesis is as follows:

\paragraph{Chapter~\ref{sec:lit_review}.} First, I provide a comprehensive background of VBDs, vector control tools, risk perception, and behavioural attitudes towards preventive measures. I broadly review mathematical and agent-based approaches to modelling VBD spread before discussing the current landscape of ABMs that integrate preventive behaviours.

\paragraph{Chapter~\ref{sec:baseline_model}.} Next, I introduce the existing ABM chosen as a baseline model to extend within this thesis. I then detail the process to replicate the model and validate my implementation by reproducing the results of the original authors.

\paragraph{Chapter~\ref{sec:extended_model}.} After defining the baseline model, I address Objectives~\ref{ro1} and \ref{ro2} by outlining the extensions made to the model and detailing the parameterisation process to recreate a real-world setting based on qualitative and quantitative survey data. I then examine the behaviour of this model by varying its parameters and observing the subsequent impacts on disease spread.

\paragraph{Chapter~\ref{sec:bcts}.} Using the extended model, I complete Objectives~\ref{ro3}--\ref{ro5} by implementing two behaviour change theories as computational decision-making processes. I simulate hypothetical scenarios and compare output across these theories to assess the impacts on preventive behaviours and disease spread. Lastly, I conduct experiments to quantify the sensitivity of the implemented theories to changes in their inputs.

\paragraph{Chapter~\ref{sec:discussion}.} Using the results from the three previous chapters, I synthesise my findings to answer the research question. I discuss the strengths and contributions of my research and I conclude by highlighting the limitations and potential directions for future work.

\paragraph{Chapter~\ref{sec:conclusion}.} Lastly, I summarise the main findings of this thesis and I explain how my results are relevant to the broader research landscape.