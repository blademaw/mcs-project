\section{Discussion}\label{sec:discussion}

In this chapter, I discuss the implications of my findings by synthesising the results from the baseline model replication in Chapter~\ref{sec:baseline_model}, the extensions made to the model in Chapter~\ref{sec:extended_model}, and the integration of behaviour change theories (BCTs) in Chapter~\ref{sec:bcts}. First, I explain how the research objectives set out in Chapter~\ref{sec:introduction} were achieved and how the findings in this thesis answer the research question. Then, I situate these results within the broader research landscape, and I cover the strengths and limitations of my approach. Finally, I provide suggestions for future work based on the outcomes of this thesis.

\subsection{Summary of findings}

The aim of this thesis was to investigate how incorporating representations of human behaviour within computational models of vector-borne diseases (VBD) could affect the dynamics of preventive measure adoption and impact disease spread. To address my research question, I extended an existing agent-based model (ABM) with preventive measures and BCTs, and I simulated hypothetical disease scenarios to assess how BCTs impact preventive behaviours and disease spread. Overall, my results indicate that while integrating BCTs into models of VBDs can lead to complex patterns of preventive measure efficacy that subsequently alter infection dynamics, the subjective representations of BCTs in such models are significant determinants of these dynamics.

Chapter~\ref{sec:baseline_model} provided the necessary modelling foundations by successfully replicating an existing agent-based model (ABM) of VBD spread. By repeating the same experiments as the original authors and achieving the expected results, I confirmed the correctness of the baseline model prior to its extension. This reproduction of a computational model of VBD spread accepted within the literature ensured the conceptual validity of the following work in Chapter~\ref{sec:extended_model}, where I extended and parameterised the baseline ABM to ground the model in a real-world setting using survey data. The purpose of this was twofold: First, to implement a mechanism for the chosen preventive measure, insecticide-treated bed nets (ITNs); and second, to contextualise the model and create \q{low-dimensional realism} by using survey data from the Mondulkiri province. This completed Objectives~\ref{ro1} and \ref{ro2}, and was a pre-requisite for integrating psychological BCTs into the model due to their basis in reality.

Through my experiments, I concluded that the extended model was able to reproduce the salient patterns observed in the relevant VBD literature: Agents were more likely to be infected during the hours of dusk, nighttime, and dawn, and disease dynamics were heterogeneous across the three provincial regions with varying forest cover. I simulated various ITN adoption levels: High levels of ITN adoption led to substantial decreases in disease transmission, although a key finding was that the impacts of ITNs depended on occupation, notably having less of an effect for forest workers. Further analysis of the impacts of ITN use on disease spread is provided in Section~\ref{sec:extended-model-discussion}.

To relax the assumption of fixed ITN use and enrich the extended model with dynamic patterns of preventive behaviours, I integrated two computational formulations of BCTs in Chapter~\ref{sec:bcts}. This satisfied Objective~\ref{ro3} of incorporating psychological theories of decision-making in the ABM, where I detailed how existing mathematical formulations of BCTs informed my implementations. Then, I simulated seasonal VBD infection risk and varied parameters across both BCTs to assess their sensitivity to changes in behavioural attitudes (Objectives~\ref{ro4}--\ref{ro5}). When comparing model output between the two BCTs, I concluded that while both implementations had similar responses to variation in behavioural attitudes, the dynamics of preventive behaviours and disease spread ultimately depended on the specific BCT employed. Finally, as discussed in Section~\ref{sec:bcts-discussion}, I observed that not only did differences in the conceptual formulation between BCTs influence model dynamics, but even small changes in the computational implementation of both BCTs could lead to large changes in model output.

\subsection{Strengths, contributions, and implications}

The first core contribution of this work is a methodological example of how BCTs can be computationally integrated into ABMs. There are still relatively few instances of ABMs that employ psychological theories of behaviour change \cite{weston_infection_2018}. Yet, the results presented in Chapter~\ref{sec:bcts} indicate that the integration of BCTs can have drastic impacts on preventive behaviours and disease spread. To the best of my knowledge, this work is not only the first to attempt to integrate BCTs into an ABM of VBD spread, but also to analyse the impacts of employing different BCTs. My findings suggest how different BCTs---in this case, the Health Belief Model (HBM) and Protection Motivation Theory (PMT)---can lead to distinct dynamics of preventive behaviours. Therefore, if modellers wish to incorporate BCTs into models of disease spread to create nuanced representations of behaviour, one crucial consideration is which BCT should be chosen for the model.

In my findings, not only was the choice of BCT important, but also its underlying formulation. As discussed above, when integrating BCTs into the extended model in Chapter~\ref{sec:bcts}, I observed how small changes in conceptual and technical BCT formulations could lead to large changes in preventive behaviour and disease dynamics. This is aligned with findings from \citet{durham_incorporating_2012}, who noted the sensitivity of the HBM formulation to changes in parameters, and also echoed by \citet{scheidegger_agent-based_2017}, who suggested that different behavioural formulations in ABMs of disease spread can lead to vastly different dynamics within the same base model. As noted by \citet{kurchyna_seeing_2024}, this sensitivity of the BCT formulations suggests the need for thorough empirical validation of the mechanisms used to derive the construct values within BCT implementations.

The second main contribution of this work is a general framework for modelling VBDs within an ABM that includes BCTs as decision-making processes for agents. While the extended model presented in Chapter~\ref{sec:extended_model} was parameterised specifically to the Mondulkiri province, the general process I follow to draw on qualitative and empirical VBD survey data could be repeated to contextualise the model to any area. Furthermore, the model presented here could be used as a test bed for health interventions: The third experiment in Section~\ref{sec:bcts-experiments} demonstrates how interventions or public health campaigns that target specific constructs in BCTs could be naturally tested for their efficacy in promoting preventive behaviours by directly varying agents' construct values at model runtime. Despite these strengths, however, there are a number of limitations to the extended model and the BCTs discussed above, with opportunities for future work which I discuss below.

\subsection{Limitations and future work}

First, I inherit the limitations from the ABM which I base my work on: As mentioned in Chapter~\ref{sec:baseline_model}, the patch model used in the baseline model from \citet{manore_network-patch_2015} assumes that heterogeneous household-household disease spread does not exist. However, as \citet{pepey_mobility_2022} noted, \q{malaria risk exposure can vary from one village to another,} meaning this assumption is not true. Another substantial limitation is that all models developed in this project simulate the disease as an outbreak, where in reality, VBDs are typically endemic diseases that fluctuate in prevalence over time \cite{bhatia_vector-borne_2014}. Future work could modify the model presented here with a re-infection mechanism that would allow recovered agents to become susceptible again at a defined rate, with a \q{burn-in} period of initial disease spread to recreate an endemic setting.

There are also conceptual limitations and barriers to integrating BCTs into ABMs. First, as \citet{durham_incorporating_2012} noted, research into how psychological BCT constructs \q{respond to external influences during an epidemic} is still sparse. This means it is not fully understood whether belief constructs have constant influences on health behaviours over an epidemic, or if they vary over time according to contextual and internal processes. Second, parameterisation remains a key barrier to integrating BCTs in ABMs, as it is not always possible to find sufficient data to empirically ground all components of behavioural mechanisms \cite{durham_incorporating_2012, scheidegger_agent-based_2017}. Lastly, encoding BCT constructs is a subjective process, as noted in Section~\ref{sec:bcts-strengths-and-limitations}, which may lead to an over-specified model or arbitrary threshold values and mechanisms. As \citet{durham_incorporating_2012} observed, this practice requires a \q{compromise between psychological realism and model tractability.}

There are also limitations specific to the extended model presented in this thesis. First, the simplified movement model assumes identical daily agent movement patterns, meaning agents repeat their typical routines even when infected (and symptomatic). Second, the distribution, cost, and quality of ITNs are not factors considered in the model, despite being important for widespread adoption \cite{manuv_investigating_2023}. Lastly, ITNs are the only preventive measure included in the model, despite the prevalence of other preventive measures such as mosquito repellent and coils \cite{phok_behavioural_2022}. Future work could incorporate the wear and tear of ITNs over time, additional preventive measures, and extend the movement model to limit agent travel in outdoor areas when perceptions of disease are high, since spending less time outdoors has been shown to be a preventive behaviour for VBD epidemics \cite{duval_how_2022}.

Future work could also extend the BCT implementations, or conduct further analysis to investigate their dynamics. First, additional mechanisms could be added to formulate VBD-specific constructs: One example of this could be agents' responsiveness to mosquito density to inform perceived susceptibility, as this is a known indicator of infection risk perceptions \cite{raude_public_2012}. Second, additional agent attributes could be added to act as inputs to BCT constructs, such as cultural practices, education levels, and financial resources \cite{watanabe_determinants_2014, naserrudin_role_2022}. Lastly, the reliance of the HBM on threshold values to trigger indicator constructs could be quantified by conducting sensitivity analysis across these thresholds.

Finally, there were challenges to computationally representing the chosen BCTs in the extended model. First, modelling social norms as cues to action in the HBM was a subjective choice, as it is unclear where the influence of social normative behaviours best fits within the HBM's constructs. Similarly in the PMT, the framing of ITN use as socially \textit{unacceptable} behaviour was counterintuitive as it meant that social pressure was not represented a driving force of adoption within the BCT. Second, in many VBD studies, perceived susceptibility and severity were categorised under the same construct \cite{kakaire_role_2023, watanabe_determinants_2014, yirsaw_insecticide-treated_2021, fonzo_we_2024, raude_public_2012}, blurring the line between these two constructs in the HBM. These two issues are indicative of the greater challenge of computationally translating conceptual theories of behaviour: Formulations of BCTs are non-rigorously defined and can evolve over time \cite{norman_protection_2015, champion_health_2015}, so any implementation in a computational model should take this into consideration.