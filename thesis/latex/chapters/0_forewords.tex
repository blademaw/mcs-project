\section*{Abstract}
\addcontentsline{toc}{section}{Abstract}

Vector-borne diseases (VBDs) such as malaria and dengue represent approximately 17\% of all infectious diseases globally. While traditional mathematical models are useful for predicting disease spread, these models often neglect individual risk perceptions and behavioural attitudes that influence preventive measure use, and consequently, infection dynamics. Computational techniques such as agent-based models (ABMs) have emerged to more accurately capture heterogeneous factors of VBD spread, yet most ABMs still employ simple representations of behaviour, despite a wealth of psychological literature on how individuals change their behaviour in disease contexts. 

Although prior work has implemented conceptual behaviour change theories (BCTs) into ABMs, the methodological implications of their integration into VBD models remain unclear, and no studies have compared the effects of implementing different BCTs. This thesis addresses this research gap by investigating how incorporating different BCTs within ABMs of VBD spread can influence the dynamics of preventive behaviours and disease spread.

In this work, I extend an ABM of VBD spread with preventive measures, and I parameterise the model to a real-world setting. Drawing on existing mathematical formulations, I implement two psychological BCTs and simulate hypothetical scenarios to assess their impacts on preventive behaviours, infection dynamics, and the model's alignment to real-world observations. I show that integrating BCTs into ABMs can lead to nuanced representations of behaviour that enable mechanistic explanations for preventive measure use, and I demonstrate how this could be useful for designing health interventions. However, I also show that different BCTs can yield distinct results within the same model, and that their computational implementations can be highly sensitive to small variations in their inputs and formulations.

Overall, this thesis illustrates the methodological implications of integrating different BCTs into models of disease spread, and contributes a foundational model of VBD spread that captures complex patterns of preventive behaviours which can inform future work.

% Vector-borne diseases (VBDs) such as malaria and dengue remain global health concerns, representing approximately 17\% of all infectious diseases. Mathematical models have emerged as useful tools to anticipate and understand the spread of such diseases, guiding policy and responses to disease outbreaks. However, these models have historically neglected important heterogeneous factors of risk perception and behavioural attitudes that influence the preventive behaviours of at-risk individuals. To address this, computational techniques such as agent-based models (ABMs) have been used to incorporate heterogenous factors of disease spread.

% Although ABMs have become widely adopted in the field of VBD modelling, most models use simple representations of human behaviour, despite a wealth of literature from psychology describing how individuals change their behaviour when at risk of disease. While prior work has integrated behaviour change theories into ABMs, the methodological implications of integrating such theories into models in VBD contexts is not clear, and little attention has been paid to the effects of using two different behaviour change theories. Evidently, there exists a gap in understanding around how computationally implementations of conceptual behaviour change theories within ABMs of VBD spread can influence the dynamics of preventive behaviours, and consequently, disease spread.

% To address this, I extend a general ABM of VBD spread with preventive measures, I parameterise the model to a real-world setting, and I implement two psychological behaviour change theories. I simulate hypothetical scenarios to assess the model's alignment to real-world observations, in addition to how incorporating behaviour change theories influences preventive behaviours and infection dynamics. I show that integrating behaviour change theories into ABMs can lead to nuanced representations of behaviour that enable mechanistic explanations for preventive measure use within at-risk populations. I demonstrate how this could be useful for designing health interventions. However, I also show that different behaviour change theories can lead to different results in the same base model, and that the computational implementations of such theories can be highly sensitive to small changes in behavioural dynamics.

% Overall, this thesis demonstrates the methodological implications of integrating behaviour change theories into models of disease spread, and . This thesis lays the groundwork of a general framework for modelling VBD spread with integrated behaviour change theories and preventive measures that can be extended in future work.

\clearpage

\section*{Declaration}
\addcontentsline{toc}{section}{Declaration}

I, Jack Oliver, on the 27th of October 2024, certify that:
\begin{itemize}
    \item this thesis does not incorporate, without acknowledgement, any material previously submitted for a degree or diploma in any university; and that to the best of my knowledge and belief it does not contain any material previously published or written by another person where due reference is not made in the text.
    \item no clearance from the University's Ethics Committee was necessary to carry out this research.
    \item the thesis is 22,923 words in length (excluding text in images, table, bibliographies and appendices).
\end{itemize}

\clearpage

\section*{Acknowledgements}
\addcontentsline{toc}{section}{Acknowledgements}

I extend my sincere gratitude to my supervisors Associate Professor Nic Geard and Dr Cameron Zachreson for their incredibly valuable and well-informed advice, surprising responsiveness to emails, witty and tasteful humour, persistent encouragement, and willingness to defend (and critique) my work. I thank them for somehow finding time each week to discuss my project while managing a multitude of other commitments. \\

\noindent I am incredibly fortunate to have been financially supported entirely throughout this degree by the Airwallex Excellence in Technology Masters Scholarship and the University of Melbourne Graduate Scholarship. I am also in a place of privilege to be supported by my parents, both personally and financially. \\

\noindent I am greatly indebted to my friends, coworkers, and loved ones for their continuous guidance, support, advice, and extended patience with me as I attempted to juggle full-time work and school.
