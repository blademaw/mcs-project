\section{Conclusion}\label{sec:conclusion}

Preventive behaviours are profoundly important for understanding the spread of vector-borne diseases (VBDs) within at-risk populations, yet representations of human behaviour in computational models of VBDs are often greatly simplified. In this thesis, I demonstrated how integrating behaviour change theories (BCTs) from psychology into agent-based models (ABMs) of VBD spread can lead to more nuanced representations of behaviour, and as a result, produce complex patterns of preventive behaviours and disease spread. I showed how integrating BCTs into these models enables mechanistic explanations for preventive behaviour dynamics, and could allow policymakers to better understand the potential impacts and efficacy of health interventions. However, I also concluded that the additional complexity of BCTs can make their formulations sensitive to small changes in behavioural attitudes, and that different BCTs can lead to distinct predictions of preventive measure adoption.

This work is significant for two main reasons: First, to the best of my knowledge, it is the earliest example of how BCTs can be integrated into a parameterised ABM of VBD spread. My results using this ABM are relevant to the broader research landscape as they demonstrate the potential methodological impacts of using different BCTs, and the consequences of uninformed formulations of BCT constructs. Second, this thesis presents a general framework for VBD spread that can be adapted to other geographical and disease contexts. While I base my model on a specific region and single preventive measure, modellers can reproduce the general approach I take in extending and parameterising an ABM prior to encoding BCTs within agents as decision-making processes.

This thesis lays a foundation for future work to extend the model presented here with more realistic endemic disease patterns, additional mechanisms that inform behavioural attitudes, alternative derivations of BCT constructs that are thoroughly grounded in empirical data, and BCTs other than the two investigated.