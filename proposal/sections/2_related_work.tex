
\section{Related work}

In this section, I review past and present literature relevant to this proposed project. First, I present an overview of VBDs in epidemiological and modelling contexts, and examine how risk perception and behaviour change theories from psychology influence preventive behaviours. Second, I cover traditional methods for VBD modelling and I motivate agent-based modelling, before concluding with recent approaches to representing human behaviour in ABMs.


\subsection{Vector-borne diseases}

VBDs are infectious diseases that are major burdens on public health systems, with over one billion infections every year and symptomatic cases regularly requiring hospitalisation \cite{world_health_organisation_who_global_2004}. VBDs are spread by \textit{vectors}, which are living organisms that enable human-human or animal-human transmission of infectious pathogens, with examples including malaria and dengue, transmitted by mosquitoes, and leishmaniasis, transmitted by sandflies \cite{world_health_organisation_who_vector-borne_2020}. Not only do these diseases put pressure on health systems, but they also increase health inequalities and hinder socioeconomic development in developing nations that have insufficient coverage of health services, or tropical climates where VBDs are more common \cite{campbell-lendrum_climate_2015}.

There are multiple determinants that affect how VBDs spread. As demonstrated in Figure~\ref{fig:transmission-diagram}, VBDs can have bidirectional infection mechanisms, in which the vector infects the susceptible host, and the host can subsequently infect an uninfected vector. These host-vector transmission dynamics can be impacted by weather effects such as temperature and humidity, which affect feeding behaviour, survival rates, and reproduction among mosquitoes \cite{campbell-lendrum_climate_2015}.  Additionally, climate effects such as precipitation, floods, and droughts can influence vector abundance, activity, and transmission seasonality \cite{woolhouse_early_2001}. Individuals' demographic and socioecological factors like age, housing conditions, working situation, and knowledge of health services can also influence risk of infection \cite{chala_emerging_2021, hasyim_social_2019}.

\begin{figure}[h]
\centering

\begin{tikzpicture}[node distance=1cm, auto,
                >=Latex, 
                every node/.append style={align=center, font=\small},
                int/.style={draw, thick}]

\newcommand{\drawagentat}[4]{
    \tikzset{shift={(#1,#2)}}
    
    \path[draw=black,fill=#3,miter limit=10.0,scale=#4] (1.6431, -0.2275).. controls (1.5055, -0.2275) and (1.3944, -0.3387) .. (1.3944, -0.4763).. controls (1.3944, -0.6138) and (1.5055, -0.725) .. (1.6431, -0.725).. controls (1.7806, -0.725) and (1.8918, -0.6138) .. (1.8918, -0.4763).. controls (1.8918, -0.3387) and (1.7806, -0.2275) .. (1.6431, -0.2275) -- cycle(1.3679, -0.7911).. controls (1.1906, -0.7911) and (1.0478, -0.934) .. (1.0478, -1.1113) -- (1.0478, -1.8838).. controls (1.0478, -1.9447) and (1.0954, -1.9923) .. (1.1562, -1.9923).. controls (1.2171, -1.9923) and (1.2647, -1.9447) .. (1.2647, -1.8838) -- (1.2647, -1.1853) -- (1.3203, -1.1853).. controls (1.3203, -1.1853) and (1.3203, -3.0057) .. (1.3203, -3.1247).. controls (1.3203, -3.2041) and (1.3864, -3.2703) .. (1.4658, -3.2703).. controls (1.5478, -3.2703) and (1.6113, -3.2041) .. (1.6113, -3.1247) -- (1.6113, -1.9976) -- (1.6722, -1.9976) -- (1.6722, -3.1247).. controls (1.6722, -3.2041) and (1.7383, -3.2703) .. (1.8177, -3.2703).. controls (1.8997, -3.2703) and (1.9632, -3.2041) .. (1.9632, -3.1247) -- (1.9632, -1.1853) -- (2.0188, -1.1853) -- (2.0188, -1.8838).. controls (2.0188, -1.9447) and (2.0664, -1.9923) .. (2.1273, -1.9923).. controls (2.1881, -1.9923) and (2.2357, -1.9447) .. (2.2357, -1.8838) -- (2.2357, -1.1086).. controls (2.2357, -0.9313) and (2.0902, -0.7885) .. (1.9156, -0.7885).. controls (1.9182, -0.7911) and (1.3679, -0.7911) .. (1.3679, -0.7911) -- cycle;

    \tikzset{shift={(-#1,-#2)}}
    }

\newcommand{\drawmosquitoat}[3]{
    \tikzset{shift={(#1,#2)}}

    \path[draw=black,fill=#3,miter limit=10.0,line width=0.0001cm,scale=.5] (1.5779, -0.1786).. controls (1.5669, -0.1812) and (1.5472, -0.1912) .. (1.5356, -0.1999).. controls (1.4706, -0.2489) and (1.4023, -0.3617) .. (1.3047, -0.5818) -- (1.2865, -0.6227) -- (1.2778, -0.6065).. controls (1.2549, -0.5642) and (1.2251, -0.5267) .. (1.1826, -0.4866).. controls (1.1259, -0.4332) and (1.1002, -0.4158) .. (1.0708, -0.411).. controls (1.0548, -0.4083) and (1.0384, -0.4131) .. (1.0244, -0.4245).. controls (1.0177, -0.4299) and (1.0057, -0.4453) .. (0.9996, -0.4566).. controls (0.9873, -0.4783) and (0.9739, -0.5204) .. (0.9651, -0.5645).. controls (0.9486, -0.6463) and (0.939, -0.7409) .. (0.9265, -0.9452).. controls (0.9241, -0.9832) and (0.9232, -1.1225) .. (0.925, -1.173).. controls (0.9279, -1.2566) and (0.9328, -1.3343) .. (0.9423, -1.4448).. controls (0.9449, -1.4755) and (0.9471, -1.5021) .. (0.9471, -1.5041).. controls (0.9471, -1.5074) and (0.9464, -1.508) .. (0.942, -1.5087).. controls (0.9095, -1.5139) and (0.8738, -1.5243) .. (0.8494, -1.5358).. controls (0.7839, -1.5665) and (0.7361, -1.6163) .. (0.715, -1.6761).. controls (0.711, -1.6871) and (0.7099, -1.6882) .. (0.7055, -1.6845).. controls (0.7007, -1.6805) and (0.6862, -1.6726) .. (0.676, -1.6685).. controls (0.5991, -1.6377) and (0.5126, -1.6836) .. (0.4672, -1.7791).. controls (0.4606, -1.7928) and (0.4511, -1.8196) .. (0.4474, -1.8343) -- (0.4459, -1.8401) -- (0.4319, -1.8353).. controls (0.3542, -1.8082) and (0.2813, -1.7632) .. (0.2143, -1.7006) -- (0.1987, -1.686) -- (0.1785, -1.7062) -- (0.1582, -1.7265) -- (0.1747, -1.7422).. controls (0.2473, -1.8116) and (0.3334, -1.8639) .. (0.423, -1.8932) -- (0.4392, -1.8984) -- (0.4399, -1.9033).. controls (0.4403, -1.906) and (0.4414, -1.9127) .. (0.4424, -1.9182).. controls (0.4433, -1.9237) and (0.4439, -1.9285) .. (0.4435, -1.9286).. controls (0.4419, -1.9296) and (0.4016, -1.9362) .. (0.3884, -1.9377).. controls (0.3537, -1.9416) and (0.2951, -1.9416) .. (0.2574, -1.9377).. controls (0.237, -1.9355) and (0.2109, -1.9312) .. (0.1889, -1.9267).. controls (0.1782, -1.9245) and (0.1692, -1.9231) .. (0.1686, -1.9237).. controls (0.1682, -1.9242) and (0.1652, -1.9357) .. (0.162, -1.9493).. controls (0.1587, -1.9629) and (0.1559, -1.975) .. (0.1556, -1.9762).. controls (0.155, -1.978) and (0.1585, -1.9793) .. (0.1747, -1.9827).. controls (0.243, -1.9975) and (0.3129, -2.0023) .. (0.3788, -1.9966).. controls (0.4038, -1.9944) and (0.4509, -1.9871) .. (0.4628, -1.9835).. controls (0.4646, -1.983) and (0.4661, -1.9846) .. (0.4684, -1.9894).. controls (0.4776, -2.0074) and (0.5007, -2.0393) .. (0.5224, -2.0641) -- (0.5364, -2.08) -- (0.4775, -2.5068).. controls (0.4113, -2.9862) and (0.4163, -2.9414) .. (0.4274, -2.9532).. controls (0.441, -2.9675) and (0.465, -2.964) .. (0.4732, -2.9464).. controls (0.4752, -2.9424) and (0.5019, -2.8017) .. (0.5545, -2.5211).. controls (0.5976, -2.2905) and (0.6332, -2.1008) .. (0.6336, -2.0995).. controls (0.6339, -2.0983) and (0.6395, -2.094) .. (0.6459, -2.09).. controls (0.6794, -2.0695) and (0.7143, -2.0409) .. (0.7365, -2.0158).. controls (0.7651, -1.9835) and (0.7814, -1.9521) .. (0.7869, -1.9195) -- (0.7886, -1.909) -- (0.7961, -1.9149).. controls (0.8149, -1.9296) and (0.8453, -1.9462) .. (0.8723, -1.9562).. controls (0.8863, -1.9614) and (0.9164, -1.9695) .. (0.9283, -1.9713).. controls (0.9319, -1.9718) and (0.9348, -1.9727) .. (0.9348, -1.9732).. controls (0.9348, -1.9737) and (0.8922, -1.9996) .. (0.8403, -2.0307) -- (0.7459, -2.0874) -- (0.7468, -2.0959).. controls (0.7474, -2.1006) and (0.7628, -2.2306) .. (0.781, -2.3846) -- (0.8142, -2.665) -- (0.7426, -2.7558).. controls (0.6836, -2.8308) and (0.6712, -2.8471) .. (0.6728, -2.8485).. controls (0.6799, -2.8551) and (0.7155, -2.882) .. (0.7165, -2.8814).. controls (0.7172, -2.881) and (0.753, -2.8359) .. (0.7961, -2.7812) -- (0.8745, -2.6817) -- (0.8414, -2.4034).. controls (0.8233, -2.2502) and (0.8083, -2.1234) .. (0.8081, -2.1213).. controls (0.8078, -2.1178) and (0.8129, -2.1145) .. (0.9173, -2.0519) -- (1.0268, -1.9861) -- (1.0268, -2.0117) -- (1.0268, -2.0372) -- (0.9308, -2.1333) -- (0.8348, -2.2294) -- (0.9377, -2.5005).. controls (1.0287, -2.7403) and (1.0405, -2.772) .. (1.0394, -2.7757).. controls (1.0387, -2.7779) and (1.0173, -2.8259) .. (0.9919, -2.882).. controls (0.9665, -2.9381) and (0.9459, -2.9843) .. (0.946, -2.9845).. controls (0.9482, -2.9866) and (0.9981, -3.0079) .. (0.9987, -3.0069).. controls (0.9992, -3.0062) and (1.0229, -2.9537) .. (1.0515, -2.8904) -- (1.1034, -2.7751) -- (1.0027, -2.5097) -- (0.9021, -2.2445) -- (0.9933, -2.1532) -- (1.0844, -2.0619) -- (1.0844, -2.0125) -- (1.0844, -1.9631) -- (1.0985, -1.9585).. controls (1.1268, -1.9493) and (1.1573, -1.9334) .. (1.1835, -1.9141).. controls (1.1983, -1.9031) and (1.2228, -1.8785) .. (1.2345, -1.8628).. controls (1.2452, -1.8483) and (1.2574, -1.8254) .. (1.2636, -1.8083) -- (1.2678, -1.7963) -- (1.2747, -1.8055).. controls (1.3045, -1.8462) and (1.3557, -1.8935) .. (1.4132, -1.933) -- (1.4268, -1.9425) -- (1.4434, -1.9809).. controls (1.5992, -2.3424) and (1.8249, -2.6913) .. (2.0853, -2.9734).. controls (2.1174, -3.0083) and (2.2421, -3.1347) .. (2.3751, -3.2675).. controls (2.4783, -3.3706) and (2.4786, -3.3707) .. (2.491, -3.3708).. controls (2.4986, -3.3708) and (2.5102, -3.3649) .. (2.5142, -3.3586).. controls (2.5196, -3.3503) and (2.5209, -3.3431) .. (2.5183, -3.3341).. controls (2.5161, -3.3262) and (2.5151, -3.3254) .. (2.3413, -3.1512).. controls (2.1781, -2.9877) and (2.1495, -2.9585) .. (2.1125, -2.9178).. controls (1.9152, -2.7011) and (1.7411, -2.4489) .. (1.5967, -2.1702).. controls (1.5676, -2.1139) and (1.5104, -1.9945) .. (1.5119, -1.993).. controls (1.5122, -1.9927) and (1.5242, -1.9983) .. (1.5389, -2.0055).. controls (1.5794, -2.025) and (1.6312, -2.0464) .. (1.6606, -2.0559) -- (1.6706, -2.059) -- (1.6876, -2.0888).. controls (1.7911, -2.2693) and (1.9248, -2.4309) .. (2.0646, -2.5443).. controls (2.145, -2.6096) and (2.2281, -2.6597) .. (2.3128, -2.6943).. controls (2.3221, -2.698) and (2.3299, -2.7009) .. (2.3301, -2.7007).. controls (2.3309, -2.6997) and (2.3494, -2.6494) .. (2.3494, -2.6483).. controls (2.3494, -2.6478) and (2.3405, -2.6436) .. (2.3295, -2.639).. controls (2.1491, -2.5639) and (1.9748, -2.4107) .. (1.8226, -2.1934).. controls (1.8011, -2.1629) and (1.7604, -2.1001) .. (1.7532, -2.0865) -- (1.7517, -2.0836) -- (1.765, -2.0869).. controls (1.8057, -2.0968) and (1.8595, -2.1058) .. (1.9026, -2.1101).. controls (1.9359, -2.1134) and (2.0011, -2.1134) .. (2.0296, -2.1101).. controls (2.1515, -2.0961) and (2.2196, -2.0457) .. (2.22, -1.9692).. controls (2.2201, -1.9444) and (2.2149, -1.9239) .. (2.2012, -1.8958).. controls (2.1855, -1.8637) and (2.1652, -1.8365) .. (2.1318, -1.8031).. controls (2.1074, -1.7787) and (2.0924, -1.7657) .. (2.0629, -1.7433).. controls (2.0249, -1.7146) and (1.9751, -1.6837) .. (1.9331, -1.6626).. controls (1.9261, -1.6591) and (1.9204, -1.6554) .. (1.9204, -1.6546).. controls (1.9204, -1.6536) and (1.9304, -1.6318) .. (1.9425, -1.606) -- (1.9646, -1.5589) -- (1.9888, -1.5908).. controls (2.1673, -1.8266) and (2.3523, -2.0214) .. (2.5062, -2.1353).. controls (2.5221, -2.1473) and (2.5168, -2.1389) .. (2.5695, -2.2321).. controls (2.6142, -2.3112) and (2.6394, -2.3502) .. (2.6764, -2.3985).. controls (2.7088, -2.4408) and (2.7379, -2.4732) .. (2.7784, -2.5118).. controls (2.8458, -2.5763) and (2.9168, -2.6241) .. (2.9957, -2.6582) -- (3.0138, -2.6662) -- (3.0143, -2.6711) -- (3.0147, -2.6762) -- (3.0255, -2.6762).. controls (3.0468, -2.6762) and (3.0565, -2.6721) .. (3.0623, -2.6606).. controls (3.0664, -2.6522) and (3.0675, -2.6444) .. (3.0655, -2.6368).. controls (3.0624, -2.6254) and (3.0553, -2.6205) .. (3.0255, -2.6083).. controls (3.004, -2.5995) and (2.9587, -2.5761) .. (2.9366, -2.5624).. controls (2.888, -2.5321) and (2.8475, -2.4997) .. (2.8023, -2.4545).. controls (2.7551, -2.4073) and (2.7194, -2.3634) .. (2.6797, -2.3037).. controls (2.6617, -2.2765) and (2.6256, -2.2163) .. (2.6267, -2.2152).. controls (2.627, -2.2149) and (2.6363, -2.2198) .. (2.6476, -2.2258).. controls (2.6897, -2.2486) and (2.7291, -2.2648) .. (2.7921, -2.285).. controls (2.9323, -2.3297) and (3.031, -2.3496) .. (3.1248, -2.3518) -- (3.1587, -2.3527) -- (3.1587, -2.3235) -- (3.1587, -2.2946) -- (3.1399, -2.2946).. controls (3.0497, -2.2945) and (2.9471, -2.2741) .. (2.804, -2.228).. controls (2.7448, -2.2089) and (2.723, -2.2004) .. (2.6837, -2.1805).. controls (2.6551, -2.1662) and (2.6186, -2.1444) .. (2.5875, -2.1231) -- (2.5648, -2.1078) -- (2.5567, -2.0938).. controls (2.4746, -1.9522) and (2.3375, -1.7313) .. (2.2032, -1.5251).. controls (2.1623, -1.4623) and (2.0375, -1.275) .. (2.0359, -1.2741).. controls (2.0352, -1.2736) and (2.0331, -1.2768) .. (2.0312, -1.2812).. controls (2.0293, -1.2855) and (2.011, -1.3244) .. (1.9907, -1.3677) -- (1.9538, -1.4462) -- (1.9303, -1.413).. controls (1.9031, -1.374) and (1.8711, -1.3265) .. (1.8489, -1.292).. controls (1.8335, -1.268) and (1.8333, -1.2676) .. (1.8317, -1.2715).. controls (1.8308, -1.2735) and (1.8043, -1.343) .. (1.7729, -1.4256).. controls (1.7413, -1.5083) and (1.7153, -1.5762) .. (1.7151, -1.5765).. controls (1.7147, -1.5769) and (1.7026, -1.5743) .. (1.688, -1.5707).. controls (1.5384, -1.5341) and (1.4055, -1.5338) .. (1.3192, -1.5702).. controls (1.2877, -1.5834) and (1.2631, -1.6018) .. (1.248, -1.6236) -- (1.2427, -1.6314) -- (1.235, -1.6201) -- (1.2275, -1.6085) -- (1.295, -1.5407).. controls (1.4109, -1.4246) and (1.4906, -1.3355) .. (1.5698, -1.234).. controls (1.7196, -1.0417) and (1.8035, -0.8709) .. (1.8263, -0.7117).. controls (1.8409, -0.6094) and (1.8294, -0.5137) .. (1.791, -0.4187).. controls (1.7723, -0.3727) and (1.7319, -0.2971) .. (1.7015, -0.2513).. controls (1.6742, -0.2102) and (1.6459, -0.1858) .. (1.6175, -0.1786).. controls (1.607, -0.176) and (1.589, -0.176) .. (1.5779, -0.1786) -- cycle(2.0642, -1.4188).. controls (2.1854, -1.6011) and (2.327, -1.8232) .. (2.4194, -1.9761) -- (2.4355, -2.0027) -- (2.4227, -1.9912).. controls (2.2985, -1.8785) and (2.162, -1.7246) .. (2.0231, -1.5404).. controls (2.0066, -1.5187) and (1.9932, -1.5003) .. (1.9932, -1.4996).. controls (1.9932, -1.4989) and (2.0028, -1.4779) .. (2.0146, -1.4528).. controls (2.0264, -1.4277) and (2.0379, -1.4031) .. (2.0401, -1.3983).. controls (2.0423, -1.3935) and (2.0447, -1.3901) .. (2.0452, -1.3908).. controls (2.0459, -1.3914) and (2.0544, -1.4041) .. (2.0642, -1.4188) -- cycle(1.8622, -1.4156).. controls (1.8703, -1.4275) and (1.8878, -1.4524) .. (1.9009, -1.4709).. controls (1.9141, -1.4895) and (1.925, -1.5052) .. (1.925, -1.5059).. controls (1.925, -1.5072) and (1.8687, -1.6284) .. (1.8673, -1.6302).. controls (1.8667, -1.6308) and (1.8636, -1.6297) .. (1.8346, -1.6178).. controls (1.8234, -1.6131) and (1.8048, -1.6063) .. (1.7934, -1.6024).. controls (1.7822, -1.5986) and (1.7723, -1.5949) .. (1.7716, -1.5943).. controls (1.7709, -1.5938) and (1.7851, -1.5548) .. (1.8031, -1.5077).. controls (1.8452, -1.3971) and (1.8464, -1.394) .. (1.847, -1.394).. controls (1.8473, -1.394) and (1.8541, -1.4038) .. (1.8622, -1.4156) -- cycle;

    \tikzset{shift={(-#1,-#2)}}
}

\node[] (u-v) at (0,0) {Uninfected\\vector};
\node[below right=2.5cm of u-v] (i-h) {Infected\\host};
\node[below left=2.5cm of i-h] (i-v) {Infected\\vector};
\node[below left=2.5cm of u-v] (u-h) {Uninfected\\host};

\coordinate[above=1.5cm of i-h] (i-h2);

\path[->,thick,bend left]
    (u-v) edge node[font=\footnotesize] {bites} (i-h2)
    (i-h) edge node[font=\footnotesize] {infects} (i-v)
    (i-v) edge node[font=\footnotesize] {bites} (u-h);
\path[->,thick] (u-h) edge node[font=\footnotesize] {becomes} (i-h);

\drawmosquitoat{-.7}{2.2}{s-color}
\drawmosquitoat{-.6}{-3.35}{i-color}
\drawagentat{2.95}{-.95}{i-color}{.4}
\drawagentat{-4.55}{-.95}{s-color}{.4}

\end{tikzpicture}

\bcaption{Human-vector disease transmission for dengue.}{Hosts are infected by infectious mosquitoes, and mosquitoes are infected by infectious hosts. Adapted from \citet{krzhizhanovskaya_agent-based_2020}.}
\label{fig:transmission-diagram}
\end{figure}

\subsubsection{Vector control methods}

Most determinants of disease transmission mentioned above can be addressed through interventions known collectively as vector control, which can be broadly classified into chemical- and non-chemical-based tools \cite{wilson_importance_2020}. For example, housing conditions that are situated close to vector breeding grounds can be treated by spraying insecticides indoors to repel mosquitoes, and long-sleeved clothing can reduce the risk of being bitten while working in harsh outdoor conditions. Until the use of the first residual insecticide dichlorodiphenyltrichloroethane (commonly known as DDT) in the 1940s, vector control was essentially synonymous with environmental management \cite{wilson_importance_2020}. Nowadays, however, most vector control methods are chemical practices, such as outdoor fogging or spraying, and insecticide-treated nets.

Despite recent advancements in vector control, limiting VBD spread is still an ongoing and challenging public health concern. Vector populations are undergoing expansion and re-emergence in many areas previously thought to be essentially eradicated of disease \cite{atkinson_global_2010}. Concurrently, vectors are becoming increasingly resistant to insecticides \cite{degroote_interventions_2018, hemingway_averting_2016}, challenging the reliance on modern-day vector control methods. While insecticide resistance is an important factor for vector control, insecticides are arguably just one component of disease control, and an over-allocation of resources toward solving insecticide resistance may detract from \q{other key factors such as \dots effectively engaging communities,} according to \citet{okumu_what_2022}. Indeed, the need for holistic strategies that engage relevant social stakeholders has been echoed by experts in the field, highlighting the need for community-based approaches for VBD control \cite{bardosh_addressing_2017}.

Community-based interventions (CBIs) are interventions that promote VBD control through communal events. CBIs cover a broad range of interventions from surveillance and risk mapping projects to clean-up programs and educational empowerment campaigns \cite{perez_realist_2021}. While CBIs have the potential to increase the rate of use and effectiveness of vector control commodities \cite{winch_effectiveness_1992}, they require timely deployment and careful consideration of a community's character.  For example, \citet{sulistyawati_dengue_2019} emphasise the need to implement \q{bottom-up} CBIs that consider community opinions about vector control, in addition to improving \q{people's knowledge and motivation to participate.} Similarly, \citet{tapia-conyer_community_2012} raise concerns about CBIs that are successful in the short-term but have waning participation in the long run, arguing that for CBI success, \q{long-term behavioural modifications at an individual level are imperative.} Furthermore, CBIs can be ineffective in communities with a lack of political authority, and in fragmented societies where community members have competing interests or a lack of trust \cite{khun_community_2008}. Evidently, the success of CBIs to encourage preventive behaviours is dependent on policymakers' understanding and planning around the behavioural aspect of the community in question.

\subsubsection{Risk perception and human behaviour}\label{sec:risk-perception}

The dynamics between risk perception and preventive behaviours are understudied relationships in infectious disease literature, despite being profoundly important for understanding disease spread, and consequently, ways of designing effective CBIs \cite{williams_role_2010}. In their seminal papers, Weinstein \textit{et al.} \cite{weinstein_correct_1993, weinstein_use_1998} formulate hypotheses for the interactions between risk perception and preventive behaviours, which were later formalised by \citet{brewer_risk_2004} into the following:

\begin{description}
\item[Accuracy hypothesis.] Perceptions of risk accurately reflect one's (lack of) preventive behaviours.
\item[Behaviour motivation hypothesis.] Risk perception causes preventive behaviours.
\item[Risk reappraisal hypothesis.] Preventive behaviours lower risk perception.
\end{description}

The causal directionalities of these hypotheses are demonstrated in Figure~\ref{fig:risk_hypotheses}.

\begin{figure}[h]
\centering

\begin{tikzpicture}[node distance=1cm, auto,
                >=Latex, 
                every node/.append style={align=center, font=\small},
                int/.style={draw, thick}]

\node (prt1) {\textbf{Perceived risk}\\at time $t$};
\node[right=5cm of prt1] (prt2) {\textbf{Perceived risk}\\at time $t+1$};
\node[below=3cm of prt1] (pbt1) {\textbf{Preventive}\\\textbf{behaviour}\\at time $t$};
\node[below=3cm of prt2] (pbt2) {\textbf{Preventive}\\\textbf{behaviour}\\at time $t+1$};

\path[->, very thick] (prt1) edge node {$+$} (prt2)
    (prt1) edge node {$+$} (pbt2)
    (pbt2) edge node [align=right,font=\itshape] {Risk\\reappraisal\\hypothesis} (prt2)
    (pbt1) edge[dashed] node {$+$} (pbt2);
\path[<->, very thick]
    (pbt2) edge[bend right, in=-120, out=-60] node {$-$} (prt2)
    (prt1) edge[dashed, bend right,left] node [align=right,font=\itshape] {Accuracy\\hypothesis} (pbt1);

\node[align=left,font=\itshape] at (1.8,-2.2) {Behaviour\\motivation\\hypothesis};
\node[align=left,font=\itshape] at (11, -2) {Accuracy\\hypothesis};
\node at (8.3, -2) {$-$};
\node at (-.45, -2) {$-$};

\end{tikzpicture}

\bcaption{Risk perception and behaviour hypotheses.}{Dashed pathways are included for completeness but not described. Straight lines indicate causal relationships, curved lines are non-causal. Signs ($+$ or $-$) indicate positive and negative influences respectively---for example, the risk reappraisal hypothesis is represented by a negative causal relationship from preventive behaviour to risk perception. Conversely, the accuracy hypothesis is represented by a negative non-causal relationship to convey that higher levels of risk perception accompany lower rates of preventive behaviour. Adapted from \citet{brewer_risk_2004}.}
\label{fig:risk_hypotheses}
\end{figure}

Although risk perception is often regarded as a salient factor influencing preventive behaviours, the empirical evidence for this relationship is mixed. On the one hand, many studies have provided evidence for the three hypotheses above \cite{brewer_risk_2004, lopes-rafegas_contribution_2023, aerts_understanding_2020, qin_exploring_2021}, and the effect of risk perception for many phenomena is commonly regarded as a significant force for motivating preventive behaviours. As \citet{tan_severe_2004} put it, \q{when people perceive a health problem as serious they will take some kind of action.} Conversely, some studies have failed to support the hypotheses, such as \citet{raude_understanding_2019} who contested the risk reappraisal hypothesis, noting it could not explain the behaviours observed in their longitudinal study of chikungunya spread. However, previous research into the driving factors of risk perception for VBDs has shown that the perceived mosquito prevalence, amount of mosquito bites on an individual, and previous days with rainfall are important indicators for individuals when estimating their risk of VBD infection \cite{raude_public_2012, lopes-rafegas_contribution_2023, constant_ecology_2020}. Furthermore, as \citet{qin_exploring_2021} and others note, the lack of empirical evidence for the linkage between risk perception and preventive behaviour may be attributable to the dominance of cross-sectional studies in VBD literature. This is because cross-sectional studies only sample a population at a single point in time, and are therefore problematic for demonstrating causality as no temporal sequence can be established. To conclude, while the above hypotheses have been evidenced by practitioners across multiple epidemiological domains, there is currently no consensus on a complete mechanistic explanation as to why people adopt preventive measures.

Psychological behaviour change theories are frameworks that can provide researchers with hypotheses for why individuals change their behaviour, and can be applied to preventive behaviours for disease spread. According to a 2024 review by \citet{vande_velde_integrated_2024}, there are 83 behaviour change theories across the behavioural and social sciences. The Health Belief Model (HBM) and Protection Motivation Theory (PMT) are two pervasive theoretical models often used in VBD studies due to their existing bodies of research and empirical support. Originally introduced in the 1950s by \citet{hochbaum_public_1958}, and later extended by \citet{becker_health_1974}, the HBM combines factors of individuals to predict why people will adopt preventive behaviours for illnesses. The framework is rooted in value-expectation theory and is based on the assumption that individuals value avoiding illness, and will practice preventive measures if they expect actions to prevent or alleviate disease \cite{champion_health_2015}. In a similar vein, PMT was introduced in the 1970s by \citet{rogers_protection_1975} and later updated \cite{rogers_cognitive_1983}. PMT formulates the intent to adopt preventive measures based on individuals' appraisals of fear---the model postulates that individuals assess the significance of a perceived threat (in this case, disease) and the available methods of coping with the threat (preventive measures) to determine the behaviour they are most inclined to perform \cite{norman_protection_2015}. The HBM and PMT are more similar than different in formation, although PMT's inclusion of some external factors in shaping behavioural intent arguably makes the framework less focused on the individual compared to the HBM.

While PMT and the HBM have been applied to VBD studies to identify endogenous and exogenous factors that motivate the adoption of preventive measures \cite{vande_velde_integrated_2024, vizanko_modeling_2024, donohoe_tick-borne_2018, ghahremani_effect_2014}, both frameworks have been criticised to varying degrees. For example, \citet{bunton_theories_1991} and others have noted that the HBM does not \q{incorporate the wider context in which social decision and actions are taken} \cite{williams_role_2010}. Similarly, PMT has been criticised for lacking a social risk component, despite multiple studies demonstrating the importance of social risk severity perceptions for predicting preventive behaviours in other health scenarios \cite{pechmann_what_2003} and in VBD epidemics \cite{lopes-rafegas_contribution_2023, vande_velde_integrated_2024}. Furthermore, both the HBM and PMT are formulated in terms of behavioural \textit{intention}, not behaviour, and do not explicitly address this disconnect from intent to realised behaviour.

To address the need for a comprehensive behaviour change theory, \citet{michie_behaviour_2011} developed the COM-B (Capability, Opportunity, Motivation, and Behaviour) model, noting that the HBM fails to \q{address the important roles of impulsivity, habit, self-control, associative learning, and emotional progressing.} The COM-B model, or simply \textit{COM-B}, requires individuals to possess the physical and physiological capability to participate, the social and physical opportunity, and the sufficient endogenous motivation to adopt a behaviour. Unlike the HBM and PMT, COM-B was designed from the ground up to include social and environmental factors that influence behaviour, and provide a framework for characterising and designing specific interventions.


\subsubsection{Modelling vector-borne diseases}\label{sec:model-vbds}

From the above discussion, vector control campaigns evidently present a complex challenge for public health practitioners: health officials must consider not only effective commodities to curb disease transmission, but also the intricate socioeconomic factors that influence human behaviour and community participation. Fortunately, mathematical models have emerged as useful tools to guide policy decisions and vector control interventions by modelling the interactions between vector and human populations. The majority of these are compartmental equation-based models (EBMs), typically defined by a system of ordinary differential equations (ODEs) that explicitly represent the population-level mechanisms of disease spread. The SIR (Susceptible, Infected, and Recovered) model is perhaps the most basic and well-known compartmental model in epidemiology, and is defined by the differential equations in Figure~\ref{fig:sir-eqs}, and diagrammatically represented in Figure~\ref{fig:sir-diagram}. In the compartmental SIR model, each compartment ($S$, $I$, or $R$) represents either the number or proportion of individuals in each state. In the latter case shown in Figure~\ref{fig:sir-fig}, the sum of compartments $S+I+R=1$, and the differential equations govern the flow between states: $\beta$ is the rate of transmission, and $1/\gamma$ is the average infectious period.

\begin{figure}[htbp]
     \centering
     \begin{subfigure}[b]{0.3\textwidth}
         \centering
         \begin{align*}
             \dot{S}&=-\beta S I \\
             \dot{I}&=\beta S I - \gamma I \\
             \dot{R}&=\gamma I
         \end{align*}
         \vspace{.5cm}
         \caption{}
         \label{fig:sir-eqs}
     \end{subfigure}%
     \hfill
     \begin{subfigure}[b]{0.3\textwidth}
         \centering
         \begin{tikzpicture}[node distance=1cm, auto,
                >=Latex, 
                every node/.append style={align=center},
                int/.style={draw, thick, minimum width=2cm,minimum height=.6cm,rounded corners}]
            
               \node [int, fill=s-color, text=white] (S)             {$S$};
               \node [int, below=of S, fill=i-color, text=white] (I) {$I$};
               \node [int, below=of I, fill=r-color, text=white] (R) {$R$};
               \coordinate[below=of I] (out);
               \path[->] (S) edge node {$\beta S I$} (I)
                         (I) edge node {$\gamma$} (out);
        \end{tikzpicture}%
        \vspace{.1cm}
         \caption{}
         \label{fig:sir-diagram}
     \end{subfigure}%
     \hfill
     \begin{subfigure}[b]{0.3\textwidth}
         \centering
         \begin{tikzpicture}[x=0.025cm,y=3cm]
      \def\xmax{130.0} % x axis maximum
      \def\ymax{1.025} % y axis maximum
      
      % CURVES
      \draw[very thick,s-color] plot[smooth] coordinates {
        (0.0, 0.998)
        (1.0, 0.9973276981931711)
        (2.0, 0.9964840728072826)
        (3.0, 0.9954259410355902)
        (4.0, 0.9940995088472758)
        (5.0, 0.9924379202498846)
        (6.0, 0.9903583301294537)
        (7.0, 0.9877584623411796)
        (8.0, 0.9845126449693982)
        (9.0, 0.9804673761499587)
        (10.0, 0.9754365737308862)
        (11.0, 0.9691968214949727)
        (12.0, 0.9614831599950985)
        (13.0, 0.9519862997172801)
        (14.0, 0.940352554204059)
        (15.0, 0.9261882782122965)
        (16.0, 0.9090710438495231)
        (17.0, 0.8885700212989189)
        (18.0, 0.8642777263133894)
        (19.0, 0.8358540642118198)
        (20.0, 0.8030810771908262)
        (21.0, 0.7659229280505638)
        (22.0, 0.7245810914666934)
        (23.0, 0.6795310390592031)
        (24.0, 0.6315262208094109)
        (25.0, 0.5815597728549975)
        (26.0, 0.530784287703032)
        (27.0, 0.480401961203948)
        (28.0, 0.4315470131997255)
        (29.0, 0.3851844173656373)
        (30.0, 0.34204311236478063)
        (31.0, 0.30259079319787735)
        (32.0, 0.26704619116159567)
        (33.0, 0.23541747963832935)
        (34.0, 0.20755337693627968)
        (35.0, 0.18319550431436243)
        (36.0, 0.16202449489961224)
        (37.0, 0.14369631108544192)
        (38.0, 0.12786819190459633)
        (39.0, 0.1142154240467327)
        (40.0, 0.10244085770765672)
        (41.0, 0.09227917117513319)
        (42.0, 0.08349762407542727)
        (43.0, 0.07589465262860504)
        (44.0, 0.06929728200938529)
        (45.0, 0.06355801522279642)
        (46.0, 0.058551620655525126)
        (47.0, 0.054172042625136994)
        (48.0, 0.050329592261200325)
        (49.0, 0.04694844269012026)
        (50.0, 0.043964461012344906)
        (51.0, 0.04132334644651637)
        (52.0, 0.038979052841522095)
        (53.0, 0.03689245949626464)
        (54.0, 0.03503025674479982)
        (55.0, 0.0333640148992419)
        (56.0, 0.03186940716069393)
        (57.0, 0.03052556194112793)
        (58.0, 0.029314522922940435)
        (59.0, 0.028220798717186017)
        (60.0, 0.027230986866564763)
        (61.0, 0.02633345947719667)
        (62.0, 0.025518099944165016)
        (63.0, 0.024776082010869227)
        (64.0, 0.024099684200702437)
        (65.0, 0.023482133047441546)
        (66.0, 0.022917471110447615)
        (67.0, 0.02240044496406817)
        (68.0, 0.021926410315262507)
        (69.0, 0.021491251208289104)
        (70.0, 0.02109131113702911)
        (71.0, 0.02072333416281497)
        (72.0, 0.0203844144502876)
        (73.0, 0.020071952916559888)
        (74.0, 0.019783619900088856)
        (75.0, 0.01951732294773462)
        (76.0, 0.019271178924476662)
        (77.0, 0.01904348983513639)
        (78.0, 0.01883272179724266)
        (79.0, 0.018637486725199447)
        (80.0, 0.018456526325907375)
        (81.0, 0.01828869811954071)
        (82.0, 0.01813296312543959)
        (83.0, 0.0179883750995347)
        (84.0, 0.017854070995352137)
        (85.0, 0.017729262582964676)
        (86.0, 0.017613229006327027)
        (87.0, 0.017505310212865357)
        (88.0, 0.017404901096874745)
        (89.0, 0.0173114462974976)
        (90.0, 0.017224435562151353)
        (91.0, 0.017143399595539257)
        (92.0, 0.01706790635057259)
        (93.0, 0.016997557715887304)
        (94.0, 0.016931986521943748)
        (95.0, 0.016870853839826203)
        (96.0, 0.016813846588184817)
        (97.0, 0.016760675318771994)
        (98.0, 0.016711072272525326)
        (99.0, 0.016664789559655245)
        (100.0, 0.016621597566351946)
        (101.0, 0.01658128346014513)
        (102.0, 0.01654364986963487)
        (103.0, 0.016508513649971912)
        (104.0, 0.01647570478322638)
        (105.0, 0.016445065355601956)
        (106.0, 0.016416448635980945)
        (107.0, 0.01638971822340957)
        (108.0, 0.01636474727181619)
        (109.0, 0.016341417776474017)
        (110.0, 0.01631961991911177)
        (111.0, 0.016299251466096593)
        (112.0, 0.01628021721595586)
        (113.0, 0.016262428496385112)
        (114.0, 0.01624580267822401)
        (115.0, 0.016230262760800425)
        (116.0, 0.016215736962871975)
        (117.0, 0.01620215835153872)
        (118.0, 0.016189464506740366)
        (119.0, 0.016177597205375163)
        (120.0, 0.016166502123166716)
        (121.0, 0.01615612857185693)
        (122.0, 0.01614642924562425)
        (123.0, 0.016137359984575585)
        (124.0, 0.016128879567973667)
        (125.0, 0.01612094950436089)
      } node[above right] at (20.0, 0.85) {$S$};
    
      \draw[very thick,i-color] plot[smooth] coordinates {
        (0.0, 0.002)
        (1.0, 0.0025118551142537743)
        (2.0, 0.0031539938777885444)
        (3.0, 0.0039591662695688465)
        (4.0, 0.004968118489808619)
        (5.0, 0.00623140960500875)
        (6.0, 0.007811562939418319)
        (7.0, 0.009785566255793595)
        (8.0, 0.012247703979930483)
        (9.0, 0.015312646690744908)
        (10.0, 0.019118631546431588)
        (11.0, 0.023830424406796634)
        (12.0, 0.029641548066158235)
        (13.0, 0.036774978103104934)
        (14.0, 0.04548115670733093)
        (15.0, 0.056031784071024576)
        (16.0, 0.06870752067429403)
        (17.0, 0.08377764007513508)
        (18.0, 0.10147010366212483)
        (19.0, 0.12193183146840725)
        (20.0, 0.14518140067848814)
        (21.0, 0.17105999575147074)
        (22.0, 0.19919043532171044)
        (23.0, 0.22895696958160838)
        (24.0, 0.259518106856763)
        (25.0, 0.28985938667094235)
        (26.0, 0.3188829532686558)
        (27.0, 0.3455194399114178)
        (28.0, 0.3688394565809396)
        (29.0, 0.3881415222032409)
        (30.0, 0.4030005812795563)
        (31.0, 0.4132730399035242)
        (32.0, 0.41906532352012577)
        (33.0, 0.42067946344958507)
        (34.0, 0.4185502560260349)
        (35.0, 0.41318559720843967)
        (36.0, 0.40511699252513234)
        (37.0, 0.39486293867085376)
        (38.0, 0.3829048359712472)
        (39.0, 0.3696734068745041)
        (40.0, 0.35554304142539545)
        (41.0, 0.34083158946059094)
        (42.0, 0.32580354734024025)
        (43.0, 0.31067510055440883)
        (44.0, 0.29561995717048395)
        (45.0, 0.28077527980532796)
        (46.0, 0.26624730330554247)
        (47.0, 0.25211645104600255)
        (48.0, 0.2384418197612011)
        (49.0, 0.22526506832660712)
        (50.0, 0.21261370867495347)
        (51.0, 0.20050387694464075)
        (52.0, 0.1889426384526646)
        (53.0, 0.17792989454199318)
        (54.0, 0.16745995095769076)
        (55.0, 0.1575228009041371)
        (56.0, 0.1481051718320928)
        (57.0, 0.13919137482870902)
        (58.0, 0.13076399175912412)
        (59.0, 0.12280442866621256)
        (60.0, 0.11529335930002195)
        (61.0, 0.10821107868483248)
        (62.0, 0.10153778306609723)
        (63.0, 0.09525378974173093)
        (64.0, 0.08933970751617225)
        (65.0, 0.08377656795680846)
        (66.0, 0.07854592336352069)
        (67.0, 0.07362991889742818)
        (68.0, 0.06901134319755721)
        (69.0, 0.06467366191877792)
        (70.0, 0.06060103750187934)
        (71.0, 0.056778337871093404)
        (72.0, 0.053191136511508266)
        (73.0, 0.04982570562316056)
        (74.0, 0.04666900390809084)
        (75.0, 0.04370866035237411)
        (76.0, 0.04093295490047132)
        (77.0, 0.03833079688115406)
        (78.0, 0.03589170191195289)
        (79.0, 0.033605767757493035)
        (80.0, 0.0314636496459622)
        (81.0, 0.029456535300379883)
        (82.0, 0.02757612011540257)
        (83.0, 0.025814582538337398)
        (84.0, 0.024164559970481298)
        (85.0, 0.022619125203910007)
        (86.0, 0.021171763595026684)
        (87.0, 0.01981635095076292)
        (88.0, 0.018547132265826043)
        (89.0, 0.017358701287442397)
        (90.0, 0.016245980943696357)
        (91.0, 0.015204204657283225)
        (92.0, 0.014228898518535435)
        (93.0, 0.013315864297351382)
        (94.0, 0.012461163329890494)
        (95.0, 0.011661101240363472)
        (96.0, 0.010912213393931912)
        (97.0, 0.0102112512067788)
        (98.0, 0.0095551691146386)
        (99.0, 0.008941112345388831)
        (100.0, 0.008366405291875594)
        (101.0, 0.007828540611958488)
        (102.0, 0.007325168898255837)
        (103.0, 0.00685408898038212)
        (104.0, 0.00641323876526587)
        (105.0, 0.00600068664013983)
        (106.0, 0.005614623366158602)
        (107.0, 0.00525335446207554)
        (108.0, 0.004915293041732614)
        (109.0, 0.004598953085528574)
        (110.0, 0.004302943118755094)
        (111.0, 0.004025960273615112)
        (112.0, 0.003766784714830889)
        (113.0, 0.003524274404150664)
        (114.0, 0.003297360197385097)
        (115.0, 0.00308504122679214)
        (116.0, 0.0028863805892277226)
        (117.0, 0.0027005012887202306)
        (118.0, 0.002526582425212876)
        (119.0, 0.0023638556598361343)
        (120.0, 0.002211601837857245)
        (121.0, 0.0020691478868978476)
        (122.0, 0.0019358638729487004)
        (123.0, 0.001811160237143461)
        (124.0, 0.0016944852515148568)
        (125.0, 0.0015853225676223741)
        } node[above right] at (45.0, 0.3) {$I$};
    
        \draw[very thick,r-color] plot[smooth] coordinates {
        (0.0, 0.0)
        (1.0, 0.00016044669257483907)
        (2.0, 0.00036193331492854905)
        (3.0, 0.0006148926948404921)
        (4.0, 0.0009323726629151103)
        (5.0, 0.001330670145106066)
        (6.0, 0.001830106931127467)
        (7.0, 0.002455971403026405)
        (8.0, 0.0032396510506710074)
        (9.0, 0.004219977159296082)
        (10.0, 0.005444794722681841)
        (11.0, 0.006972754098230427)
        (12.0, 0.00887529193874307)
        (13.0, 0.011238722179614701)
        (14.0, 0.014166289088609818)
        (15.0, 0.017779937716678768)
        (16.0, 0.022221435476182713)
        (17.0, 0.027652338625945735)
        (18.0, 0.034252170024485565)
        (19.0, 0.0422141043197727)
        (20.0, 0.05173752213068529)
        (21.0, 0.06301707619796504)
        (22.0, 0.07622847321159593)
        (23.0, 0.09151199135918818)
        (24.0, 0.10895567233382564)
        (25.0, 0.12858084047405985)
        (26.0, 0.1503327590283119)
        (27.0, 0.17407859888463387)
        (28.0, 0.1996135302193346)
        (29.0, 0.22667406043112157)
        (30.0, 0.2549563063556628)
        (31.0, 0.2841361668985982)
        (32.0, 0.31388848531827834)
        (33.0, 0.34390305691208545)
        (34.0, 0.3738963670376853)
        (35.0, 0.40361889847719784)
        (36.0, 0.4328585125752554)
        (37.0, 0.46144075024370435)
        (38.0, 0.48922697212415656)
        (39.0, 0.5161111690787633)
        (40.0, 0.542016100866948)
        (41.0, 0.566889239364276)
        (42.0, 0.5906988285843326)
        (43.0, 0.6134302468169863)
        (44.0, 0.635082760820131)
        (45.0, 0.6556667049718758)
        (46.0, 0.6752010760389326)
        (47.0, 0.6937115063288607)
        (48.0, 0.7112285879775989)
        (49.0, 0.7277864889832728)
        (50.0, 0.7434218303127018)
        (51.0, 0.7581727766088432)
        (52.0, 0.7720783087058135)
        (53.0, 0.7851776459617423)
        (54.0, 0.7975097922975096)
        (55.0, 0.8091131841966213)
        (56.0, 0.8200254210072134)
        (57.0, 0.8302830632301633)
        (58.0, 0.8399214853179355)
        (59.0, 0.8489747726166015)
        (60.0, 0.8574756538334134)
        (61.0, 0.865455461837971)
        (62.0, 0.8729441169897378)
        (63.0, 0.8799701282473998)
        (64.0, 0.8865606082831254)
        (65.0, 0.89274129899575)
        (66.0, 0.8985366055260318)
        (67.0, 0.9039696361385037)
        (68.0, 0.9090622464871803)
        (69.0, 0.913835086872933)
        (70.0, 0.9183076513610916)
        (71.0, 0.9224983279660917)
        (72.0, 0.9264244490382042)
        (73.0, 0.9301023414602796)
        (74.0, 0.9335473761918203)
        (75.0, 0.9367740166998912)
        (76.0, 0.939795866175052)
        (77.0, 0.9426257132837095)
        (78.0, 0.9452755762908044)
        (79.0, 0.9477567455173074)
        (80.0, 0.9500798240281304)
        (81.0, 0.9522547665800793)
        (82.0, 0.9542909167591578)
        (83.0, 0.9561970423621279)
        (84.0, 0.9579813690341665)
        (85.0, 0.9596516122131252)
        (86.0, 0.9612150073986462)
        (87.0, 0.9626783388363717)
        (88.0, 0.9640479666372991)
        (89.0, 0.9653298524150598)
        (90.0, 0.9665295834941521)
        (91.0, 0.9676523957471773)
        (92.0, 0.9687031951308919)
        (93.0, 0.9696865779867614)
        (94.0, 0.9706068501481657)
        (95.0, 0.9714680449198103)
        (96.0, 0.9722739400178834)
        (97.0, 0.9730280734744492)
        (98.0, 0.9737337586128361)
        (99.0, 0.9743940980949559)
        (100.0, 0.9750119971417724)
        (101.0, 0.9755901759278963)
        (102.0, 0.9761311812321092)
        (103.0, 0.9766373973696458)
        (104.0, 0.9771110564515075)
        (105.0, 0.977554248004258)
        (106.0, 0.9779689279978604)
        (107.0, 0.9783569273145148)
        (108.0, 0.9787199596864512)
        (109.0, 0.9790596291379974)
        (110.0, 0.9793774369621331)
        (111.0, 0.9796747882602882)
        (112.0, 0.9799529980692132)
        (113.0, 0.9802132970994641)
        (114.0, 0.9804568371243909)
        (115.0, 0.9806846960124074)
        (116.0, 0.9808978824479002)
        (117.0, 0.9810973403597408)
        (118.0, 0.9812839530680466)
        (119.0, 0.9814585471347885)
        (120.0, 0.981621896038976)
        (121.0, 0.9817747235412451)
        (122.0, 0.9819177068814269)
        (123.0, 0.9820514797782809)
        (124.0, 0.9821766351805113)
        (125.0, 0.9822937279280166)
        } node[above right] at (75.0, 0.775) {$R$};

        % AXES
      \draw[<->,thick]
        (\xmax,0) node[below left] {$t$}
        -| (0,\ymax) node[above left,rotate=90] {Proportion};
    \end{tikzpicture}%
    \vspace{.05cm}
         \caption{}
         \label{fig:sir-curve}
     \end{subfigure}%
        \bcaption{Compartmental SIR model.}{\textbf{(i)} System of differential equations representing changes in flows between compartments. \textbf{(ii)} Compartmental diagram representing flows between compartments. \textbf{(iii)} SIR dynamics over time for $\beta=0.3$ and $1/\gamma=14$.}
        \label{fig:sir-fig}
\end{figure}

Compartmental models that extend the classical SIR model have long been used in epidemiological and VBD contexts \cite{tang_review_2020}. In his review, \citet{cosner_models_2015} covered spatial and non-spatial EBMs for VBDs, noting that \q{even simple models can provide insights into how movement patterns can affect disease transmission.} More complex EBMs have incorporated vector control measures such as drugs and insecticide spraying to analyse the efficacy and timeliness of these interventions \cite{reiner_systematic_2013}. Despite the success of these models, however, they are limited in that populations are assumed to be homogeneous, meaning infection rates and other factors such as mosquito biting rates are constant, despite studies suggesting otherwise \cite{perkins_heterogeneity_2013}. During their review of 388 equation-based VBD models from 1970 to 2010, \citet{reiner_systematic_2013} noted that most modern VBD models focused solely on \q{basic transmission dynamics} closely resembling the original Ross-Macdonald malaria model \cite{ross_report_1908,macdonald_epidemiology_1957} and claimed the field could benefit from models that incorporate heterogeneous factors.

There have been research efforts to facilitate heterogeneity in EBMs, despite the assumption of homogeneous populations. For example, \citet{perkins_heterogeneity_2013} proposed a framework for heterogeneous biting, transmission thresholds, and spatial scales of transmission by creating multiple \q{patches} or a \textit{metapopulation} that represented groups of populations characterised by different systems of equations. \citet{suarez_generic_2020} also used a patch-based architecture to develop an EBM that incorporated varied risk perception and demand for mosquito control. Similarly, \citet{roosa_general_2022} combined a patch model with an extended compartmental SIR model to represent the adoption of preventive measures within an EBM. Ultimately, however, ODE models are inherently incompatible with heterogeneity---specifying the dynamics of disease spread and behaviour at the population level means heterogeneity can only be captured via patches, yet as \citet{bedson_review_2021} point out, \q{the history of major epidemics demonstrates that \textit{microscale interactions between individuals} [emphasis added] are important.}

% ABMs

\subsection{Agent-based modelling}

The inability of EBMs to incorporate heterogeneity at the individual level outlined above is one reason why modellers have recently turned to the use of agent-based models (ABMs). Contrary to the top-down specification EBMs require, ABMs define a population of decentralised, autonomous agents that interact with one another to reproduce or \q{grow} emergent phenomena \cite{epstein_growing_1996}. Unlike the fixed system structure of EBMs, ABMs typically consist of simple rules that can be readily changed to alter system mechanisms. This \textit{bottom-up} approach to modelling naturally allows for the heterogeneity and adaptation of agents \cite{axtell_agent-based_2022}, and greatly simplifies the representation of interventions or counterfactual scenarios \cite{van_dyke_parunak_agent-based_1998}. In the context of epidemiology, an ABM can thus \q{function as an \textit{in silico} laboratory} for health practitioners to simulate populations encoded with heterogeneous biological, spatial, and behavioural characteristics \cite{marshall_formalizing_2015}.

A corollary of the malleable structure of ABMs is that existing models can be easily extended by other practitioners with alternative goals. The discrete agent-based representation of the SIR model illustrated in Figure~\ref{sir-abm-diag} is one example: $N$ agents encoded with the SIR states randomly move within an $N_b\times N_b$ lattice. If an infected agent and a susceptible agent occupy the same cell in any given time step, the susceptible agent is infected with probability $\beta$. Similar to the equation-based form, after an average $1/\gamma$ time steps, infected agents recover. Since ABMs are stochastic, simulations are often repeated and results are averaged over multiple runs, as shown in Figure~\ref{fig:sir-abm-avg}. The simple SIR model has been extended by, e.g., \citet{paoluzzi_single-agent_2021}, who alter agent mobility to investigate the impacts of restrictions during the COVID-19 epidemic.

\begin{figure}[h]
\centering

\begin{tikzpicture}[scale=1.1]
    \def\ioffset{0.525} % x axis maximum
    \def\matsize{5}
    
    % Draw labels
    \foreach \i in {0,1}
        \path (\i+\ioffset,-1+\ioffset) node{\i} (-1+\ioffset,\i+\ioffset) node{\i};
    \path (2+\ioffset,-1+\ioffset) node{$\dots$} (-1+\ioffset,2+\ioffset) node[rotate=90] {$\dots$};
    \path (3+\ioffset,-1+\ioffset) node{$i$} (-1+\ioffset,3+\ioffset) node{$j$};
    \path (4+\ioffset,-1+\ioffset) node{$\dots$} (-1+\ioffset,4+\ioffset) node[rotate=90] {$\dots$};
    \path (\matsize+\ioffset,-1+\ioffset) node{$N_b$} (-1+\ioffset,\matsize+\ioffset) node{$N_b$};
      
    % Draw lattice rects
    \foreach \i in {0,...,\matsize}
        \foreach \j in {0,...,\matsize}{
            % Check if current cell is (1,2) or (3,4) and color green
            \ifthenelse{\(\i=0 \AND \j=0\) \OR \(\i=0 \AND \j=2\) \OR \(\i=0 \AND \j=3\) \OR \(\i=0 \AND \j=4\) \OR \(\i=1 \AND \j=0\) \OR \(\i=1 \AND \j=1\) \OR \(\i=1 \AND \j=2\) \OR \(\i=2 \AND \j=0\) \OR \(\i=2 \AND \j=1\) \OR \(\i=2 \AND \j=2\) \OR \(\i=2 \AND \j=4\) \OR \(\i=3 \AND \j=1\) \OR \(\i=3 \AND \j=3\) \OR \(\i=3 \AND \j=4\) \OR \(\i=4 \AND \j=0\) \OR \(\i=4 \AND \j=4\) \OR \(\i=4 \AND \j=5\) \OR \(\i=5 \AND \j=0\) \OR \(\i=5 \AND \j=3\) \OR \(\i=5 \AND \j=4\)}{
                \fill[s-color] (\i+0.1,\j+0.1) rectangle (\i+0.95,\j+0.95);
            }{}

            \ifthenelse{\(\i=3 \AND \j=0\) \OR \(\i=4 \AND \j=3\) \OR \(\i=4 \AND \j=2\) \OR \(\i=2 \AND \j=5\)}{
                \fill[i-color] (\i+0.1,\j+0.1) rectangle (\i+0.95,\j+0.95);
            }{}

            \ifthenelse{\(\i=1 \AND \j=5\) \OR \(\i=5 \AND \j=5\) \OR \(\i=4 \AND \j=1\)}{
                \fill[r-color] (\i+0.1,\j+0.1) rectangle (\i+0.95,\j+0.95);
            }{}

            
            \draw[thick] (\i+0.1,\j+0.1) rectangle (\i+0.95,\j+0.95);
        };

    % Draw agents
    % R agents
    % \node[person,shirt=r-color,scale=1.2] at (1.1+\ioffset,5.3+\ioffset) {};
    % \node[person,shirt=r-color,scale=1.2] at (1-.1+\ioffset,5+\ioffset) {};
    
    % S agents
    % \node[person,shirt=s-color,scale=1.2] at (1-.1+\ioffset,1-.1+\ioffset) {};
    
    % \node[person,shirt=s-color,scale=1.2] at (2.1+\ioffset,1.3+\ioffset) {};
    % \node[person,shirt=s-color,scale=1.2] at (2-.1+\ioffset,1+\ioffset) {};

    % I agents
    % \node[black,scale=1.1] at (3-.1+\ioffset,0.1+\ioffset) {\usymH{1F6B9}{.9cm}};
    % \node[i-color] at (3-.1+\ioffset,0.1+\ioffset) {\usymH{1F6B9}{.8cm}};
    % \node[i-color] at (3-.1+\ioffset,0.1+\ioffset) {};

    \node[isosceles triangle, isosceles triangle apex angle=90, fill=s-color, minimum size=.66cm, rotate=225] at (4.38,2.37) {};
    \draw[thick] (4+0.1,2+0.1) rectangle (4+0.95,2+0.95);

    \newcommand{\drawagent}[1]{
    \path[draw=black,fill=#1,miter limit=10.0,scale=.3] (1.6431, -0.2275).. controls (1.5055, -0.2275) and (1.3944, -0.3387) .. (1.3944, -0.4763).. controls (1.3944, -0.6138) and (1.5055, -0.725) .. (1.6431, -0.725).. controls (1.7806, -0.725) and (1.8918, -0.6138) .. (1.8918, -0.4763).. controls (1.8918, -0.3387) and (1.7806, -0.2275) .. (1.6431, -0.2275) -- cycle(1.3679, -0.7911).. controls (1.1906, -0.7911) and (1.0478, -0.934) .. (1.0478, -1.1113) -- (1.0478, -1.8838).. controls (1.0478, -1.9447) and (1.0954, -1.9923) .. (1.1562, -1.9923).. controls (1.2171, -1.9923) and (1.2647, -1.9447) .. (1.2647, -1.8838) -- (1.2647, -1.1853) -- (1.3203, -1.1853).. controls (1.3203, -1.1853) and (1.3203, -3.0057) .. (1.3203, -3.1247).. controls (1.3203, -3.2041) and (1.3864, -3.2703) .. (1.4658, -3.2703).. controls (1.5478, -3.2703) and (1.6113, -3.2041) .. (1.6113, -3.1247) -- (1.6113, -1.9976) -- (1.6722, -1.9976) -- (1.6722, -3.1247).. controls (1.6722, -3.2041) and (1.7383, -3.2703) .. (1.8177, -3.2703).. controls (1.8997, -3.2703) and (1.9632, -3.2041) .. (1.9632, -3.1247) -- (1.9632, -1.1853) -- (2.0188, -1.1853) -- (2.0188, -1.8838).. controls (2.0188, -1.9447) and (2.0664, -1.9923) .. (2.1273, -1.9923).. controls (2.1881, -1.9923) and (2.2357, -1.9447) .. (2.2357, -1.8838) -- (2.2357, -1.1086).. controls (2.2357, -0.9313) and (2.0902, -0.7885) .. (1.9156, -0.7885).. controls (1.9182, -0.7911) and (1.3679, -0.7911) .. (1.3679, -0.7911) -- cycle;
    }

    \newcommand{\drawagentat}[3]{
    \tikzset{shift={(#2,#3)}}
    
    \path[draw=black,fill=#1,miter limit=10.0,scale=.3] (1.6431, -0.2275).. controls (1.5055, -0.2275) and (1.3944, -0.3387) .. (1.3944, -0.4763).. controls (1.3944, -0.6138) and (1.5055, -0.725) .. (1.6431, -0.725).. controls (1.7806, -0.725) and (1.8918, -0.6138) .. (1.8918, -0.4763).. controls (1.8918, -0.3387) and (1.7806, -0.2275) .. (1.6431, -0.2275) -- cycle(1.3679, -0.7911).. controls (1.1906, -0.7911) and (1.0478, -0.934) .. (1.0478, -1.1113) -- (1.0478, -1.8838).. controls (1.0478, -1.9447) and (1.0954, -1.9923) .. (1.1562, -1.9923).. controls (1.2171, -1.9923) and (1.2647, -1.9447) .. (1.2647, -1.8838) -- (1.2647, -1.1853) -- (1.3203, -1.1853).. controls (1.3203, -1.1853) and (1.3203, -3.0057) .. (1.3203, -3.1247).. controls (1.3203, -3.2041) and (1.3864, -3.2703) .. (1.4658, -3.2703).. controls (1.5478, -3.2703) and (1.6113, -3.2041) .. (1.6113, -3.1247) -- (1.6113, -1.9976) -- (1.6722, -1.9976) -- (1.6722, -3.1247).. controls (1.6722, -3.2041) and (1.7383, -3.2703) .. (1.8177, -3.2703).. controls (1.8997, -3.2703) and (1.9632, -3.2041) .. (1.9632, -3.1247) -- (1.9632, -1.1853) -- (2.0188, -1.1853) -- (2.0188, -1.8838).. controls (2.0188, -1.9447) and (2.0664, -1.9923) .. (2.1273, -1.9923).. controls (2.1881, -1.9923) and (2.2357, -1.9447) .. (2.2357, -1.8838) -- (2.2357, -1.1086).. controls (2.2357, -0.9313) and (2.0902, -0.7885) .. (1.9156, -0.7885).. controls (1.9182, -0.7911) and (1.3679, -0.7911) .. (1.3679, -0.7911) -- cycle;

    \tikzset{shift={(-#2,-#3)}}
    }

    % S agents
    \tikzset{shift={(.4+\ioffset,1.7+\ioffset)}}
        \drawagent{s-color}
    \tikzset{shift={(1.25,.25)}}
        \drawagent{s-color}
    \tikzset{shift={(-.35,-.25)}}
        \drawagent{s-color}

    % R agents
    \tikzset{shift={(-.95,4.2)}}
        \drawagent{r-color}
    \tikzset{shift={(.3,-.1)}}
        \drawagent{r-color}
    \tikzset{shift={(-.2,-.15)}}
        \drawagent{r-color}

    % I agent
    \tikzset{shift={(1.85,-4.7)}}
        \drawagent{i-color}

    % S and I agent
    \tikzset{shift={(1.4,2.05)}}
        \drawagent{i-color}
    \tikzset{shift={(-.35,-.23)}}
        \drawagent{s-color}

    % Equations
    \tikzset{shift={(3.25, 1.8)}}
    \node[] at (0,0) {\textbf{(a)}};
    \drawagentat{s-color}{0.2}{-.1}
    \drawagentat{i-color}{0.6}{-.8}
    \path[->, draw, thick] (1.8, -1) -- (3, -1);
    \node at (2.4, -.7) {$\beta$};
    
    \drawagentat{i-color}{3.1}{-.1}
    \drawagentat{i-color}{3.5}{-.8}

    \tikzset{shift={(0, -3)}}
    \node[] at (0,0) {\textbf{(b)}};
    \drawagentat{i-color}{0.4}{-.1}
    \path[->, draw, thick] (1.6, -.5) -- (3, -.5);
    \node at (2.2, -.2) {$\gamma$};
    \drawagentat{r-color}{3.1}{-.1}
    
    
\end{tikzpicture}

\bcaption{Example ABM of SIR dynamics on a lattice.}{Agents with different SIR states move randomly from any cell to any orthogonally adjacent cell. Teal agents are susceptible, red agents are infected, and purple agents are recovered. Not all agents are shown. (Multi-)coloured cells correspond to the SIR state of agents within that cell, and white cells are unoccupied. \textbf{(a)} Susceptible agents that share a cell with infected agents are infected with probability $\beta$ during each time step. \textbf{(b)} Infected agents recover after an average $1/\gamma$ time steps. Adapted from \citet{paoluzzi_single-agent_2021}.}
\label{sir-abm-diag}
\end{figure}

\begin{figure}[htbp]
    \centering

    \begin{tikzpicture}[>=stealth]
    % LHS plot 1
    \draw[curve,s-color] plot[smooth] coordinates {
    (-8.0, 3.49)
    (-7.724024512482009, 3.3899999999999997)
    (-7.6484637396372115, 3.3200000000000003)
    (-7.575293451467849, 3.26)
    (-7.503363558513278, 3.17)
    (-7.431975571153575, 3.1)
    (-7.4034740477674585, 3.04)
    (-7.347587243310602, 2.9699999999999998)
    (-7.319165044453064, 2.92)
    (-7.295761688316618, 2.85)
    (-7.256915998988681, 2.79)
    (-7.227168918566119, 2.73)
    (-7.214826797701855, 2.6399999999999997)
    (-7.190112056349427, 2.56)
    (-7.145799388697578, 2.51)
    (-7.114025393756335, 2.43)
    (-7.083940841055886, 2.4)
    (-7.065913148865518, 2.34)
    (-7.040632834972005, 2.2800000000000002)
    (-7.020523178328933, 2.21)
    (-6.994840986408573, 2.13)
    (-6.964917234604183, 2.07)
    (-6.935920958726876, 2.01)
    (-6.90240456136456, 1.96)
    (-6.871291715913282, 1.92)
    (-6.8479224101567615, 1.87)
    (-6.818213573307265, 1.83)
    (-6.776110138898, 1.78)
    (-6.700132590103333, 1.75)
    (-6.658815900705981, 1.74)
    (-6.57460720665217, 1.7)
    (-6.529765752224526, 1.68)
    (-6.45546739165365, 1.67)
    (-6.410288267674974, 1.65)
    (-6.332084009436284, 1.62)
    (-6.292160628893704, 1.59)
    (-6.164451622325407, 1.57)
    (-5.948722216423396, 1.57)
    (-5.5, 1.57)
    };
    
    \draw[curve,i-color] plot[smooth] coordinates {
    (-8.0, 1.51)
    (-7.724024512482009, 1.61)
    (-7.6484637396372115, 1.65)
    (-7.575293451467849, 1.67)
    (-7.503363558513278, 1.75)
    (-7.431975571153575, 1.79)
    (-7.4034740477674585, 1.81)
    (-7.347587243310602, 1.85)
    (-7.319165044453064, 1.85)
    (-7.295761688316618, 1.8900000000000001)
    (-7.256915998988681, 1.91)
    (-7.227168918566119, 1.93)
    (-7.214826797701855, 2.01)
    (-7.190112056349427, 2.07)
    (-7.145799388697578, 2.07)
    (-7.114025393756335, 2.13)
    (-7.083940841055886, 2.09)
    (-7.065913148865518, 2.11)
    (-7.040632834972005, 2.13)
    (-7.020523178328933, 2.17)
    (-6.994840986408573, 2.23)
    (-6.964917234604183, 2.25)
    (-6.935920958726876, 2.27)
    (-6.90240456136456, 2.27)
    (-6.871291715913282, 2.25)
    (-6.8479224101567615, 2.25)
    (-6.818213573307265, 2.23)
    (-6.776110138898, 2.23)
    (-6.700132590103333, 2.19)
    (-6.658815900705981, 2.11)
    (-6.57460720665217, 2.09)
    (-6.529765752224526, 2.0300000000000002)
    (-6.45546739165365, 1.95)
    (-6.410288267674974, 1.8900000000000001)
    (-6.332084009436284, 1.85)
    (-6.292160628893704, 1.81)
    (-6.164451622325407, 1.75)
    (-5.948722216423396, 1.65)
    (-5.5, 1.55)
    };
    
    \draw[curve,r-color] plot[smooth] coordinates {
    (-8.0, 1.5)
    (-7.724024512482009, 1.5)
    (-7.6484637396372115, 1.53)
    (-7.575293451467849, 1.57)
    (-7.503363558513278, 1.58)
    (-7.431975571153575, 1.61)
    (-7.4034740477674585, 1.65)
    (-7.347587243310602, 1.68)
    (-7.319165044453064, 1.73)
    (-7.295761688316618, 1.76)
    (-7.256915998988681, 1.8)
    (-7.227168918566119, 1.84)
    (-7.214826797701855, 1.85)
    (-7.190112056349427, 1.87)
    (-7.145799388697578, 1.92)
    (-7.114025393756335, 1.94)
    (-7.083940841055886, 2.01)
    (-7.065913148865518, 2.05)
    (-7.040632834972005, 2.09)
    (-7.020523178328933, 2.12)
    (-6.994840986408573, 2.14)
    (-6.964917234604183, 2.18)
    (-6.935920958726876, 2.2199999999999998)
    (-6.90240456136456, 2.27)
    (-6.871291715913282, 2.33)
    (-6.8479224101567615, 2.38)
    (-6.818213573307265, 2.44)
    (-6.776110138898, 2.49)
    (-6.700132590103333, 2.56)
    (-6.658815900705981, 2.65)
    (-6.57460720665217, 2.71)
    (-6.529765752224526, 2.79)
    (-6.45546739165365, 2.88)
    (-6.410288267674974, 2.96)
    (-6.332084009436284, 3.0300000000000002)
    (-6.292160628893704, 3.1)
    (-6.164451622325407, 3.1799999999999997)
    (-5.948722216423396, 3.2800000000000002)
    (-5.5, 3.38)
    };
    \draw[->,thick] (-8, 1.5) -- (-5.5, 1.5) node[below left] {$t$};
    \draw[->,thick] (-8, 1.5) -- (-8, 3.5) node[left] {$N$};

    % Dots sep
    \node[rotate=90] at (-7, 0.5) {$\dots$};

    % LHS plot 2
    \draw[curve,s-color] plot[smooth] coordinates {
    (-8.0, -0.010000000000000009)
    (-7.675905625901207, -0.10000000000000009)
    (-7.621177873639785, -0.1399999999999999)
    (-7.489215140697653, -0.17999999999999994)
    (-7.313191941569358, -0.26)
    (-7.265868612764226, -0.3500000000000001)
    (-7.2125195166257425, -0.41999999999999993)
    (-7.178492874106199, -0.47)
    (-7.1128509571699565, -0.54)
    (-7.056309963409815, -0.6100000000000001)
    (-7.026302877482503, -0.6799999999999999)
    (-7.007564697872866, -0.74)
    (-6.98289200110431, -0.81)
    (-6.9625713352402165, -0.8899999999999999)
    (-6.942163035482007, -0.97)
    (-6.9188938759677345, -1.01)
    (-6.90205091717434, -1.08)
    (-6.882410313609648, -1.13)
    (-6.82859443430481, -1.19)
    (-6.808009913481963, -1.25)
    (-6.789774262276251, -1.3199999999999998)
    (-6.7655252416613365, -1.3900000000000001)
    (-6.749582998055055, -1.4300000000000002)
    (-6.694128785342196, -1.48)
    (-6.656313058529874, -1.51)
    (-6.636879919955421, -1.52)
    (-6.597415012880587, -1.56)
    (-6.576119689438654, -1.58)
    (-6.536477553994184, -1.6099999999999999)
    (-6.492760121430862, -1.67)
    (-6.441313105638984, -1.68)
    (-6.381648323533211, -1.69)
    (-6.28967941046716, -1.69)
    (-6.160565722289649, -1.72)
    (-5.901152655592379, -1.75)
    (-5.5, -1.77)
    };
    
    \draw[curve,i-color] plot[smooth] coordinates {
    (-8.0, -1.99)
    (-7.675905625901207, -1.91)
    (-7.621177873639785, -1.93)
    (-7.489215140697653, -1.95)
    (-7.313191941569358, -1.89)
    (-7.265868612764226, -1.81)
    (-7.2125195166257425, -1.77)
    (-7.178492874106199, -1.77)
    (-7.1128509571699565, -1.73)
    (-7.056309963409815, -1.69)
    (-7.026302877482503, -1.65)
    (-7.007564697872866, -1.63)
    (-6.98289200110431, -1.59)
    (-6.9625713352402165, -1.53)
    (-6.942163035482007, -1.47)
    (-6.9188938759677345, -1.49)
    (-6.90205091717434, -1.45)
    (-6.882410313609648, -1.45)
    (-6.82859443430481, -1.4300000000000002)
    (-6.808009913481963, -1.4100000000000001)
    (-6.789774262276251, -1.37)
    (-6.7655252416613365, -1.33)
    (-6.749582998055055, -1.35)
    (-6.694128785342196, -1.35)
    (-6.656313058529874, -1.3900000000000001)
    (-6.636879919955421, -1.47)
    (-6.597415012880587, -1.49)
    (-6.576119689438654, -1.55)
    (-6.536477553994184, -1.59)
    (-6.492760121430862, -1.57)
    (-6.441313105638984, -1.65)
    (-6.381648323533211, -1.73)
    (-6.28967941046716, -1.83)
    (-6.160565722289649, -1.87)
    (-5.901152655592379, -1.91)
    (-5.5, -1.97)
    };
    
    \draw[curve,r-color] plot[smooth] coordinates {
    (-8.0, -2.0)
    (-7.675905625901207, -1.99)
    (-7.621177873639785, -1.93)
    (-7.489215140697653, -1.87)
    (-7.313191941569358, -1.85)
    (-7.265868612764226, -1.84)
    (-7.2125195166257425, -1.81)
    (-7.178492874106199, -1.76)
    (-7.1128509571699565, -1.73)
    (-7.056309963409815, -1.7)
    (-7.026302877482503, -1.67)
    (-7.007564697872866, -1.63)
    (-6.98289200110431, -1.6)
    (-6.9625713352402165, -1.58)
    (-6.942163035482007, -1.56)
    (-6.9188938759677345, -1.5)
    (-6.90205091717434, -1.47)
    (-6.882410313609648, -1.42)
    (-6.82859443430481, -1.38)
    (-6.808009913481963, -1.3399999999999999)
    (-6.789774262276251, -1.31)
    (-6.7655252416613365, -1.28)
    (-6.749582998055055, -1.22)
    (-6.694128785342196, -1.17)
    (-6.656313058529874, -1.1)
    (-6.636879919955421, -1.01)
    (-6.597415012880587, -0.95)
    (-6.576119689438654, -0.8700000000000001)
    (-6.536477553994184, -0.8)
    (-6.492760121430862, -0.76)
    (-6.441313105638984, -0.6699999999999999)
    (-6.381648323533211, -0.5800000000000001)
    (-6.28967941046716, -0.48)
    (-6.160565722289649, -0.4099999999999999)
    (-5.901152655592379, -0.3400000000000001)
    (-5.5, -0.26)
    };
    \draw[->,thick] (-8, -2) -- (-5.5, -2) node[below left] {$t$};
    \draw[->,thick] (-8, -2) -- (-8, 0) node[left] {$N$};

    % Right side plot

    \newcommand{\xshift}{-.65}
  \newcommand{\yshift}{-.5}
  \newcommand{\newyscale}{.3}
  \newcommand{\newxscale}{.25}

  % \scalebox{.25}[.3]{

  \tikzset{shift={(\xshift,\yshift)}}

  \begin{scope}[line cap=butt,line join=round,xscale=\newxscale,yscale=\newyscale]
    \begin{scope}[line cap=butt,line join=round]
      \path[fill=white,line cap=butt,line join=round] (0.0, 0.0) -- (16.256, 0.0) -- (16.256, 12.192) -- (0.0, 12.192) -- cycle;



    \end{scope}
    \begin{scope}[line cap=butt,line join=round]
      \begin{scope}[line cap=butt,line join=round]
        \begin{scope}[line cap=butt,line join=round]
          \path[draw=s-color,fill=s-color,fill opacity=0.1,draw opacity=0.1,line cap=butt,line join=round,shift={(0.0, -12.192)}] (2.6047, 22.4942) -- (2.6047, 21.3285) -- (2.7097, 21.3269) -- (2.8148, 21.3253) -- (2.9199, 21.3213) -- (3.025, 21.3173) -- (3.13, 21.311) -- (3.2351, 21.3022) -- (3.3402, 21.2927) -- (3.4452, 21.2871) -- (3.5503, 21.2752) -- (3.6554, 21.2584) -- (3.7605, 21.2441) -- (3.8655, 21.2194) -- (3.9706, 21.1916) -- (4.0757, 21.1462) -- (4.1808, 21.1048) -- (4.2858, 21.0515) -- (4.3909, 20.9862) -- (4.496, 20.9154) -- (4.6011, 20.839) -- (4.7061, 20.7466) -- (4.8112, 20.6408) -- (4.9163, 20.4959) -- (5.0214, 20.3375) -- (5.1264, 20.1831) -- (5.2315, 19.9762) -- (5.3366, 19.743) -- (5.4417, 19.4914) -- (5.5467, 19.232) -- (5.6518, 18.9685) -- (5.7569, 18.6716) -- (5.862, 18.3524) -- (5.967, 18.0237) -- (6.0721, 17.7006) -- (6.1772, 17.4116) -- (6.2823, 17.1187) -- (6.3873, 16.8155) -- (6.4924, 16.5512) -- (6.5975, 16.2519) -- (6.7025, 15.9964) -- (6.8076, 15.7712) -- (6.9127, 15.5817) -- (7.0178, 15.3955) -- (7.1228, 15.2562) -- (7.2279, 15.1097) -- (7.333, 14.9959) -- (7.4381, 14.898) -- (7.5431, 14.8073) -- (7.6482, 14.7102) -- (7.7533, 14.6401) -- (7.8584, 14.5844) -- (7.9634, 14.5383) -- (8.0685, 14.5016) -- (8.1736, 14.4722) -- (8.2787, 14.4491) -- (8.3837, 14.4252) -- (8.4888, 14.4037) -- (8.5939, 14.3807) -- (8.699, 14.3616) -- (8.804, 14.3512) -- (8.9091, 14.3345) -- (9.0142, 14.3273) -- (9.1193, 14.3218) -- (9.2243, 14.3146) -- (9.3294, 14.3082) -- (9.4345, 14.2971) -- (9.5396, 14.2883) -- (9.6446, 14.2835) -- (9.7497, 14.2788) -- (9.8548, 14.278) -- (9.9599, 14.2756) -- (10.0649, 14.2724) -- (10.17, 14.2708) -- (10.2751, 14.2668) -- (10.3801, 14.2636) -- (10.4852, 14.2621) -- (10.5903, 14.2605) -- (10.6954, 14.2581) -- (10.8004, 14.2565) -- (10.9055, 14.2565) -- (11.0106, 14.2549) -- (11.1157, 14.2533) -- (11.2207, 14.2533) -- (11.3258, 14.2525) -- (11.4309, 14.2517) -- (11.536, 14.2509) -- (11.641, 14.2509) -- (11.7461, 14.2509) -- (11.8512, 14.2493) -- (11.9563, 14.2493) -- (12.0613, 14.2477) -- (12.1664, 14.2469) -- (12.2715, 14.2461) -- (12.3766, 14.2453) -- (12.4816, 14.2453) -- (12.5867, 14.2445) -- (12.6918, 14.2445) -- (12.7969, 14.2445) -- (12.9019, 14.2438) -- (13.007, 14.2438) -- (13.1121, 14.2438) -- (13.2172, 14.2438) -- (13.3222, 14.2438) -- (13.4273, 14.2438) -- (13.5324, 14.2438) -- (13.6374, 14.2438) -- (13.7425, 14.243) -- (13.8476, 14.243) -- (13.9527, 14.2414) -- (14.0577, 14.2414) -- (14.0577, 15.4071) -- (14.0577, 15.4071) -- (13.9527, 15.4071) -- (13.8476, 15.4087) -- (13.7425, 15.4087) -- (13.6374, 15.4095) -- (13.5324, 15.4095) -- (13.4273, 15.4095) -- (13.3222, 15.4095) -- (13.2172, 15.4095) -- (13.1121, 15.4095) -- (13.007, 15.4095) -- (12.9019, 15.4095) -- (12.7969, 15.4103) -- (12.6918, 15.4103) -- (12.5867, 15.4103) -- (12.4816, 15.4111) -- (12.3766, 15.4111) -- (12.2715, 15.4119) -- (12.1664, 15.4127) -- (12.0613, 15.4135) -- (11.9563, 15.4151) -- (11.8512, 15.4151) -- (11.7461, 15.4167) -- (11.641, 15.4167) -- (11.536, 15.4167) -- (11.4309, 15.4175) -- (11.3258, 15.4183) -- (11.2207, 15.4191) -- (11.1157, 15.4191) -- (11.0106, 15.4206) -- (10.9055, 15.4222) -- (10.8004, 15.4222) -- (10.6954, 15.4238) -- (10.5903, 15.4262) -- (10.4852, 15.4278) -- (10.3801, 15.4294) -- (10.2751, 15.4326) -- (10.17, 15.4366) -- (10.0649, 15.4382) -- (9.9599, 15.4413) -- (9.8548, 15.4437) -- (9.7497, 15.4445) -- (9.6446, 15.4493) -- (9.5396, 15.4541) -- (9.4345, 15.4628) -- (9.3294, 15.474) -- (9.2243, 15.4803) -- (9.1193, 15.4875) -- (9.0142, 15.4931) -- (8.9091, 15.5002) -- (8.804, 15.517) -- (8.699, 15.5273) -- (8.5939, 15.5464) -- (8.4888, 15.5695) -- (8.3837, 15.591) -- (8.2787, 15.6149) -- (8.1736, 15.6379) -- (8.0685, 15.6674) -- (7.9634, 15.704) -- (7.8584, 15.7502) -- (7.7533, 15.8059) -- (7.6482, 15.8759) -- (7.5431, 15.973) -- (7.4381, 16.0638) -- (7.333, 16.1617) -- (7.2279, 16.2755) -- (7.1228, 16.422) -- (7.0178, 16.5612) -- (6.9127, 16.7475) -- (6.8076, 16.9369) -- (6.7025, 17.1622) -- (6.5975, 17.4177) -- (6.4924, 17.717) -- (6.3873, 17.9812) -- (6.2823, 18.2845) -- (6.1772, 18.5774) -- (6.0721, 18.8663) -- (5.967, 19.1895) -- (5.862, 19.5182) -- (5.7569, 19.8374) -- (5.6518, 20.1343) -- (5.5467, 20.3977) -- (5.4417, 20.6572) -- (5.3366, 20.9087) -- (5.2315, 21.1419) -- (5.1264, 21.3489) -- (5.0214, 21.5033) -- (4.9163, 21.6617) -- (4.8112, 21.8065) -- (4.7061, 21.9124) -- (4.6011, 22.0047) -- (4.496, 22.0811) -- (4.3909, 22.152) -- (4.2858, 22.2173) -- (4.1808, 22.2706) -- (4.0757, 22.312) -- (3.9706, 22.3573) -- (3.8655, 22.3852) -- (3.7605, 22.4099) -- (3.6554, 22.4242) -- (3.5503, 22.4409) -- (3.4452, 22.4529) -- (3.3402, 22.4584) -- (3.2351, 22.468) -- (3.13, 22.4767) -- (3.025, 22.4831) -- (2.9199, 22.4871) -- (2.8148, 22.4911) -- (2.7097, 22.4926) -- (2.6047, 22.4942) -- cycle;



        \end{scope}
      \end{scope}
      \begin{scope}[line cap=butt,line join=round]
        \begin{scope}[line cap=butt,line join=round]
          \path[draw=i-color,fill=i-color,fill opacity=0.1,draw opacity=0.1,line cap=butt,line join=round,shift={(0.0, -12.192)}] (2.6047, 14.5507) -- (2.6047, 13.9598) -- (2.7097, 13.9598) -- (2.8148, 13.9598) -- (2.9199, 13.9598) -- (3.025, 13.9598) -- (3.13, 13.9598) -- (3.2351, 13.9598) -- (3.3402, 13.9598) -- (3.4452, 13.9598) -- (3.5503, 13.9598) -- (3.6554, 13.9598) -- (3.7605, 13.9598) -- (3.8655, 13.9598) -- (3.9706, 13.9598) -- (4.0757, 13.9598) -- (4.1808, 13.9598) -- (4.2858, 13.9598) -- (4.3909, 13.9598) -- (4.496, 13.9598) -- (4.6011, 13.9598) -- (4.7061, 13.9598) -- (4.8112, 13.9598) -- (4.9163, 13.9986) -- (5.0214, 14.118) -- (5.1264, 14.2286) -- (5.2315, 14.387) -- (5.3366, 14.5621) -- (5.4417, 14.7372) -- (5.5467, 14.8964) -- (5.6518, 15.0604) -- (5.7569, 15.2347) -- (5.862, 15.4265) -- (5.967, 15.6168) -- (6.0721, 15.8062) -- (6.1772, 15.9439) -- (6.2823, 16.0641) -- (6.3873, 16.1731) -- (6.4924, 16.2559) -- (6.5975, 16.3546) -- (6.7025, 16.4095) -- (6.8076, 16.431) -- (6.9127, 16.4047) -- (7.0178, 16.3777) -- (7.1228, 16.3076) -- (7.2279, 16.2647) -- (7.333, 16.1906) -- (7.4381, 16.1007) -- (7.5431, 16.0275) -- (7.6482, 15.9503) -- (7.7533, 15.8261) -- (7.8584, 15.7194) -- (7.9634, 15.6223) -- (8.0685, 15.4966) -- (8.1736, 15.3788) -- (8.2787, 15.2697) -- (8.3837, 15.1599) -- (8.4888, 15.0628) -- (8.5939, 14.9649) -- (8.699, 14.8813) -- (8.804, 14.7874) -- (8.9091, 14.7245) -- (9.0142, 14.6338) -- (9.1193, 14.5581) -- (9.2243, 14.4833) -- (9.3294, 14.4021) -- (9.4345, 14.348) -- (9.5396, 14.2828) -- (9.6446, 14.2382) -- (9.7497, 14.1833) -- (9.8548, 14.1283) -- (9.9599, 14.0869) -- (10.0649, 14.044) -- (10.17, 14.0002) -- (10.2751, 13.9612) -- (10.3801, 13.9598) -- (10.4852, 13.9598) -- (10.5903, 13.9598) -- (10.6954, 13.9598) -- (10.8004, 13.9598) -- (10.9055, 13.9598) -- (11.0106, 13.9598) -- (11.1157, 13.9598) -- (11.2207, 13.9598) -- (11.3258, 13.9598) -- (11.4309, 13.9598) -- (11.536, 13.9598) -- (11.641, 13.9598) -- (11.7461, 13.9598) -- (11.8512, 13.9598) -- (11.9563, 13.9598) -- (12.0613, 13.9598) -- (12.1664, 13.9598) -- (12.2715, 13.9598) -- (12.3766, 13.9598) -- (12.4816, 13.9598) -- (12.5867, 13.9598) -- (12.6918, 13.9598) -- (12.7969, 13.9598) -- (12.9019, 13.9598) -- (13.007, 13.9598) -- (13.1121, 13.9598) -- (13.2172, 13.9598) -- (13.3222, 13.9598) -- (13.4273, 13.9598) -- (13.5324, 13.9598) -- (13.6374, 13.9598) -- (13.7425, 13.9598) -- (13.8476, 13.9598) -- (13.9527, 13.9598) -- (14.0577, 13.9598) -- (14.0577, 14.5865) -- (14.0577, 14.5865) -- (13.9527, 14.5929) -- (13.8476, 14.5976) -- (13.7425, 14.6016) -- (13.6374, 14.6048) -- (13.5324, 14.6096) -- (13.4273, 14.6151) -- (13.3222, 14.6215) -- (13.2172, 14.6279) -- (13.1121, 14.6311) -- (13.007, 14.635) -- (12.9019, 14.6422) -- (12.7969, 14.6518) -- (12.6918, 14.6605) -- (12.5867, 14.6685) -- (12.4816, 14.6772) -- (12.3766, 14.6852) -- (12.2715, 14.6947) -- (12.1664, 14.7019) -- (12.0613, 14.7154) -- (11.9563, 14.7226) -- (11.8512, 14.7361) -- (11.7461, 14.7513) -- (11.641, 14.7616) -- (11.536, 14.7791) -- (11.4309, 14.7911) -- (11.3258, 14.8125) -- (11.2207, 14.8388) -- (11.1157, 14.8667) -- (11.0106, 14.8953) -- (10.9055, 14.9256) -- (10.8004, 14.9566) -- (10.6954, 14.9861) -- (10.5903, 15.0155) -- (10.4852, 15.0505) -- (10.3801, 15.0887) -- (10.2751, 15.1269) -- (10.17, 15.1659) -- (10.0649, 15.2097) -- (9.9599, 15.2527) -- (9.8548, 15.2941) -- (9.7497, 15.349) -- (9.6446, 15.4039) -- (9.5396, 15.4485) -- (9.4345, 15.5138) -- (9.3294, 15.5679) -- (9.2243, 15.6491) -- (9.1193, 15.7239) -- (9.0142, 15.7995) -- (8.9091, 15.8903) -- (8.804, 15.9531) -- (8.699, 16.0471) -- (8.5939, 16.1306) -- (8.4888, 16.2285) -- (8.3837, 16.3256) -- (8.2787, 16.4355) -- (8.1736, 16.5445) -- (8.0685, 16.6623) -- (7.9634, 16.7881) -- (7.8584, 16.8852) -- (7.7533, 16.9918) -- (7.6482, 17.116) -- (7.5431, 17.1932) -- (7.4381, 17.2665) -- (7.333, 17.3564) -- (7.2279, 17.4304) -- (7.1228, 17.4734) -- (7.0178, 17.5434) -- (6.9127, 17.5705) -- (6.8076, 17.5968) -- (6.7025, 17.5753) -- (6.5975, 17.5204) -- (6.4924, 17.4217) -- (6.3873, 17.3389) -- (6.2823, 17.2298) -- (6.1772, 17.1097) -- (6.0721, 16.972) -- (5.967, 16.7825) -- (5.862, 16.5923) -- (5.7569, 16.4005) -- (5.6518, 16.2261) -- (5.5467, 16.0622) -- (5.4417, 15.903) -- (5.3366, 15.7279) -- (5.2315, 15.5528) -- (5.1264, 15.3944) -- (5.0214, 15.2837) -- (4.9163, 15.1644) -- (4.8112, 15.0497) -- (4.7061, 14.9757) -- (4.6011, 14.9112) -- (4.496, 14.8579) -- (4.3909, 14.8062) -- (4.2858, 14.7544) -- (4.1808, 14.717) -- (4.0757, 14.686) -- (3.9706, 14.6541) -- (3.8655, 14.6319) -- (3.7605, 14.612) -- (3.6554, 14.6032) -- (3.5503, 14.5905) -- (3.4452, 14.5849) -- (3.3402, 14.5817) -- (3.2351, 14.5746) -- (3.13, 14.5666) -- (3.025, 14.5602) -- (2.9199, 14.557) -- (2.8148, 14.5531) -- (2.7097, 14.5523) -- (2.6047, 14.5507) -- cycle;



        \end{scope}
      \end{scope}
      \begin{scope}[line cap=butt,line join=round]
        \begin{scope}[line cap=butt,line join=round]
          \path[draw=r-color,fill=r-color,fill opacity=0.1,draw opacity=0.1,line cap=butt,line join=round,shift={(0.0, -12.192)}] (2.6047, 14.5427) -- (2.6047, 13.9598) -- (2.7097, 13.9598) -- (2.8148, 13.9598) -- (2.9199, 13.9598) -- (3.025, 13.9598) -- (3.13, 13.9598) -- (3.2351, 13.9598) -- (3.3402, 13.9598) -- (3.4452, 13.9598) -- (3.5503, 13.9598) -- (3.6554, 13.9598) -- (3.7605, 13.9598) -- (3.8655, 13.9598) -- (3.9706, 13.9598) -- (4.0757, 13.9598) -- (4.1808, 13.9598) -- (4.2858, 13.9598) -- (4.3909, 13.9598) -- (4.496, 13.9598) -- (4.6011, 13.9598) -- (4.7061, 13.9598) -- (4.8112, 13.9598) -- (4.9163, 13.9598) -- (5.0214, 13.9598) -- (5.1264, 13.9598) -- (5.2315, 13.9598) -- (5.3366, 13.9598) -- (5.4417, 13.9598) -- (5.5467, 13.962) -- (5.6518, 14.0615) -- (5.7569, 14.1841) -- (5.862, 14.3114) -- (5.967, 14.4499) -- (6.0721, 14.5836) -- (6.1772, 14.7349) -- (6.2823, 14.9076) -- (6.3873, 15.1018) -- (6.4924, 15.2833) -- (6.5975, 15.4838) -- (6.7025, 15.6844) -- (6.8076, 15.8882) -- (6.9127, 16.1039) -- (7.0178, 16.3172) -- (7.1228, 16.5265) -- (7.2279, 16.716) -- (7.333, 16.9038) -- (7.4381, 17.0917) -- (7.5431, 17.2556) -- (7.6482, 17.4299) -- (7.7533, 17.6241) -- (7.8584, 17.7865) -- (7.9634, 17.9298) -- (8.0685, 18.0922) -- (8.1736, 18.2394) -- (8.2787, 18.3715) -- (8.3837, 18.5053) -- (8.4888, 18.6239) -- (8.5939, 18.7448) -- (8.699, 18.8475) -- (8.804, 18.9518) -- (8.9091, 19.0314) -- (9.0142, 19.1293) -- (9.1193, 19.2105) -- (9.2243, 19.2925) -- (9.3294, 19.38) -- (9.4345, 19.4453) -- (9.5396, 19.5193) -- (9.6446, 19.5686) -- (9.7497, 19.6283) -- (9.8548, 19.6841) -- (9.9599, 19.7278) -- (10.0649, 19.774) -- (10.17, 19.8194) -- (10.2751, 19.8623) -- (10.3801, 19.9037) -- (10.4852, 19.9435) -- (10.5903, 19.9801) -- (10.6954, 20.012) -- (10.8004, 20.043) -- (10.9055, 20.0741) -- (11.0106, 20.1059) -- (11.1157, 20.1362) -- (11.2207, 20.164) -- (11.3258, 20.1911) -- (11.4309, 20.2134) -- (11.536, 20.2261) -- (11.641, 20.2436) -- (11.7461, 20.254) -- (11.8512, 20.2707) -- (11.9563, 20.2842) -- (12.0613, 20.293) -- (12.1664, 20.3073) -- (12.2715, 20.3152) -- (12.3766, 20.3256) -- (12.4816, 20.3336) -- (12.5867, 20.3431) -- (12.6918, 20.3511) -- (12.7969, 20.3598) -- (12.9019, 20.3702) -- (13.007, 20.3773) -- (13.1121, 20.3813) -- (13.2172, 20.3845) -- (13.3222, 20.3909) -- (13.4273, 20.3972) -- (13.5324, 20.4028) -- (13.6374, 20.4076) -- (13.7425, 20.4116) -- (13.8476, 20.4155) -- (13.9527, 20.4219) -- (14.0577, 20.4283) -- (14.0577, 21.594) -- (14.0577, 21.594) -- (13.9527, 21.5877) -- (13.8476, 21.5813) -- (13.7425, 21.5773) -- (13.6374, 21.5733) -- (13.5324, 21.5686) -- (13.4273, 21.563) -- (13.3222, 21.5566) -- (13.2172, 21.5502) -- (13.1121, 21.5471) -- (13.007, 21.5431) -- (12.9019, 21.5359) -- (12.7969, 21.5256) -- (12.6918, 21.5168) -- (12.5867, 21.5089) -- (12.4816, 21.4993) -- (12.3766, 21.4913) -- (12.2715, 21.481) -- (12.1664, 21.473) -- (12.0613, 21.4587) -- (11.9563, 21.45) -- (11.8512, 21.4364) -- (11.7461, 21.4197) -- (11.641, 21.4094) -- (11.536, 21.3919) -- (11.4309, 21.3791) -- (11.3258, 21.3568) -- (11.2207, 21.3298) -- (11.1157, 21.3019) -- (11.0106, 21.2717) -- (10.9055, 21.2398) -- (10.8004, 21.2088) -- (10.6954, 21.1777) -- (10.5903, 21.1459) -- (10.4852, 21.1093) -- (10.3801, 21.0695) -- (10.2751, 21.0281) -- (10.17, 20.9851) -- (10.0649, 20.9398) -- (9.9599, 20.8936) -- (9.8548, 20.8498) -- (9.7497, 20.7941) -- (9.6446, 20.7344) -- (9.5396, 20.685) -- (9.4345, 20.611) -- (9.3294, 20.5458) -- (9.2243, 20.4582) -- (9.1193, 20.3762) -- (9.0142, 20.295) -- (8.9091, 20.1971) -- (8.804, 20.1175) -- (8.699, 20.0133) -- (8.5939, 19.9106) -- (8.4888, 19.7896) -- (8.3837, 19.671) -- (8.2787, 19.5373) -- (8.1736, 19.4052) -- (8.0685, 19.2579) -- (7.9634, 19.0955) -- (7.8584, 18.9523) -- (7.7533, 18.7899) -- (7.6482, 18.5957) -- (7.5431, 18.4214) -- (7.4381, 18.2574) -- (7.333, 18.0696) -- (7.2279, 17.8817) -- (7.1228, 17.6923) -- (7.0178, 17.4829) -- (6.9127, 17.2696) -- (6.8076, 17.0539) -- (6.7025, 16.8502) -- (6.5975, 16.6496) -- (6.4924, 16.449) -- (6.3873, 16.2675) -- (6.2823, 16.0733) -- (6.1772, 15.9006) -- (6.0721, 15.7494) -- (5.967, 15.6157) -- (5.862, 15.4772) -- (5.7569, 15.3498) -- (5.6518, 15.2272) -- (5.5467, 15.1277) -- (5.4417, 15.0274) -- (5.3366, 14.951) -- (5.2315, 14.8929) -- (5.1264, 14.8444) -- (5.0214, 14.8006) -- (4.9163, 14.7616) -- (4.8112, 14.7314) -- (4.7061, 14.6995) -- (4.6011, 14.6717) -- (4.496, 14.6486) -- (4.3909, 14.6295) -- (4.2858, 14.6159) -- (4.1808, 14.6) -- (4.0757, 14.5897) -- (3.9706, 14.5761) -- (3.8655, 14.5706) -- (3.7605, 14.5658) -- (3.6554, 14.5602) -- (3.5503, 14.5562) -- (3.4452, 14.5499) -- (3.3402, 14.5475) -- (3.2351, 14.5451) -- (3.13, 14.5443) -- (3.025, 14.5443) -- (2.9199, 14.5435) -- (2.8148, 14.5435) -- (2.7097, 14.5427) -- (2.6047, 14.5427) -- cycle;



        \end{scope}
      \end{scope}
      \begin{scope}[line cap=butt,line join=round]
        \path[draw=s-color,line cap=,line join=round,line width=0.05cm] (2.6047, 9.7194) -- (2.7097, 9.7178) -- (2.8148, 9.7162) -- (2.9199, 9.7122) -- (3.025, 9.7082) -- (3.13, 9.7019) -- (3.2351, 9.6931) -- (3.3402, 9.6835) -- (3.4452, 9.678) -- (3.5503, 9.666) -- (3.6554, 9.6493) -- (3.7605, 9.635) -- (3.8655, 9.6103) -- (3.9706, 9.5825) -- (4.0757, 9.5371) -- (4.1808, 9.4957) -- (4.2858, 9.4424) -- (4.3909, 9.3771) -- (4.496, 9.3063) -- (4.6011, 9.2299) -- (4.7061, 9.1375) -- (4.8112, 9.0317) -- (4.9163, 8.8868) -- (5.0214, 8.7284) -- (5.1264, 8.574) -- (5.2315, 8.367) -- (5.3366, 8.1338) -- (5.4417, 7.8823) -- (5.5467, 7.6228) -- (5.6518, 7.3594) -- (5.7569, 7.0625) -- (5.862, 6.7433) -- (5.967, 6.4146) -- (6.0721, 6.0914) -- (6.1772, 5.8025) -- (6.2823, 5.5096) -- (6.3873, 5.2063) -- (6.4924, 4.9421) -- (6.5975, 4.6428) -- (6.7025, 4.3873) -- (6.8076, 4.1621) -- (6.9127, 3.9726) -- (7.0178, 3.7864) -- (7.1228, 3.6471) -- (7.2279, 3.5006) -- (7.333, 3.3868) -- (7.4381, 3.2889) -- (7.5431, 3.1982) -- (7.6482, 3.1011) -- (7.7533, 3.031) -- (7.8584, 2.9753) -- (7.9634, 2.9291) -- (8.0685, 2.8925) -- (8.1736, 2.8631) -- (8.2787, 2.84) -- (8.3837, 2.8161) -- (8.4888, 2.7946) -- (8.5939, 2.7715) -- (8.699, 2.7524) -- (8.804, 2.7421) -- (8.9091, 2.7254) -- (9.0142, 2.7182) -- (9.1193, 2.7126) -- (9.2243, 2.7055) -- (9.3294, 2.6991) -- (9.4345, 2.688) -- (9.5396, 2.6792) -- (9.6446, 2.6744) -- (9.7497, 2.6696) -- (9.8548, 2.6689) -- (9.9599, 2.6665) -- (10.0649, 2.6633) -- (10.17, 2.6617) -- (10.2751, 2.6577) -- (10.3801, 2.6545) -- (10.4852, 2.6529) -- (10.5903, 2.6513) -- (10.6954, 2.649) -- (10.8004, 2.6474) -- (10.9055, 2.6474) -- (11.0106, 2.6458) -- (11.1157, 2.6442) -- (11.2207, 2.6442) -- (11.3258, 2.6434) -- (11.4309, 2.6426) -- (11.536, 2.6418) -- (11.641, 2.6418) -- (11.7461, 2.6418) -- (11.8512, 2.6402) -- (11.9563, 2.6402) -- (12.0613, 2.6386) -- (12.1664, 2.6378) -- (12.2715, 2.637) -- (12.3766, 2.6362) -- (12.4816, 2.6362) -- (12.5867, 2.6354) -- (12.6918, 2.6354) -- (12.7969, 2.6354) -- (12.9019, 2.6346) -- (13.007, 2.6346) -- (13.1121, 2.6346) -- (13.2172, 2.6346) -- (13.3222, 2.6346) -- (13.4273, 2.6346) -- (13.5324, 2.6346) -- (13.6374, 2.6346) -- (13.7425, 2.6338) -- (13.8476, 2.6338) -- (13.9527, 2.6322) -- (14.0577, 2.6322);



      \end{scope}
      \begin{scope}[line cap=butt,line join=round]
        \path[draw=i-color,line cap=,line join=round,line width=0.05cm] (2.6047, 1.7758) -- (2.7097, 1.7774) -- (2.8148, 1.7782) -- (2.9199, 1.7822) -- (3.025, 1.7854) -- (3.13, 1.7917) -- (3.2351, 1.7997) -- (3.3402, 1.8068) -- (3.4452, 1.81) -- (3.5503, 1.8156) -- (3.6554, 1.8283) -- (3.7605, 1.8371) -- (3.8655, 1.857) -- (3.9706, 1.8793) -- (4.0757, 1.9111) -- (4.1808, 1.9422) -- (4.2858, 1.9796) -- (4.3909, 2.0313) -- (4.496, 2.083) -- (4.6011, 2.1364) -- (4.7061, 2.2008) -- (4.8112, 2.2749) -- (4.9163, 2.3895) -- (5.0214, 2.5089) -- (5.1264, 2.6195) -- (5.2315, 2.7779) -- (5.3366, 2.953) -- (5.4417, 3.1281) -- (5.5467, 3.2873) -- (5.6518, 3.4513) -- (5.7569, 3.6256) -- (5.862, 3.8174) -- (5.967, 4.0076) -- (6.0721, 4.1971) -- (6.1772, 4.3348) -- (6.2823, 4.455) -- (6.3873, 4.564) -- (6.4924, 4.6468) -- (6.5975, 4.7455) -- (6.7025, 4.8004) -- (6.8076, 4.8219) -- (6.9127, 4.7956) -- (7.0178, 4.7686) -- (7.1228, 4.6985) -- (7.2279, 4.6555) -- (7.333, 4.5815) -- (7.4381, 4.4916) -- (7.5431, 4.4183) -- (7.6482, 4.3411) -- (7.7533, 4.217) -- (7.8584, 4.1103) -- (7.9634, 4.0132) -- (8.0685, 3.8875) -- (8.1736, 3.7696) -- (8.2787, 3.6606) -- (8.3837, 3.5508) -- (8.4888, 3.4537) -- (8.5939, 3.3558) -- (8.699, 3.2722) -- (8.804, 3.1783) -- (8.9091, 3.1154) -- (9.0142, 3.0246) -- (9.1193, 2.949) -- (9.2243, 2.8742) -- (9.3294, 2.793) -- (9.4345, 2.7389) -- (9.5396, 2.6736) -- (9.6446, 2.6291) -- (9.7497, 2.5741) -- (9.8548, 2.5192) -- (9.9599, 2.4778) -- (10.0649, 2.4348) -- (10.17, 2.3911) -- (10.2751, 2.3521) -- (10.3801, 2.3139) -- (10.4852, 2.2757) -- (10.5903, 2.2406) -- (10.6954, 2.2112) -- (10.8004, 2.1817) -- (10.9055, 2.1507) -- (11.0106, 2.1204) -- (11.1157, 2.0918) -- (11.2207, 2.0639) -- (11.3258, 2.0377) -- (11.4309, 2.0162) -- (11.536, 2.0042) -- (11.641, 1.9867) -- (11.7461, 1.9764) -- (11.8512, 1.9613) -- (11.9563, 1.9477) -- (12.0613, 1.9406) -- (12.1664, 1.927) -- (12.2715, 1.9199) -- (12.3766, 1.9103) -- (12.4816, 1.9024) -- (12.5867, 1.8936) -- (12.6918, 1.8856) -- (12.7969, 1.8769) -- (12.9019, 1.8673) -- (13.007, 1.8602) -- (13.1121, 1.8562) -- (13.2172, 1.853) -- (13.3222, 1.8466) -- (13.4273, 1.8403) -- (13.5324, 1.8347) -- (13.6374, 1.8299) -- (13.7425, 1.8267) -- (13.8476, 1.8228) -- (13.9527, 1.818) -- (14.0577, 1.8116);



      \end{scope}
      \begin{scope}[line cap=butt,line join=round]
        \path[draw=r-color,line cap=,line join=round,line width=0.05cm] (2.6047, 1.7678) -- (2.7097, 1.7678) -- (2.8148, 1.7686) -- (2.9199, 1.7686) -- (3.025, 1.7694) -- (3.13, 1.7694) -- (3.2351, 1.7702) -- (3.3402, 1.7726) -- (3.4452, 1.775) -- (3.5503, 1.7814) -- (3.6554, 1.7854) -- (3.7605, 1.7909) -- (3.8655, 1.7957) -- (3.9706, 1.8013) -- (4.0757, 1.8148) -- (4.1808, 1.8251) -- (4.2858, 1.8411) -- (4.3909, 1.8546) -- (4.496, 1.8737) -- (4.6011, 1.8968) -- (4.7061, 1.9246) -- (4.8112, 1.9565) -- (4.9163, 1.9867) -- (5.0214, 2.0257) -- (5.1264, 2.0695) -- (5.2315, 2.1181) -- (5.3366, 2.1762) -- (5.4417, 2.2526) -- (5.5467, 2.3529) -- (5.6518, 2.4524) -- (5.7569, 2.5749) -- (5.862, 2.7023) -- (5.967, 2.8408) -- (6.0721, 2.9745) -- (6.1772, 3.1257) -- (6.2823, 3.2984) -- (6.3873, 3.4927) -- (6.4924, 3.6741) -- (6.5975, 3.8747) -- (6.7025, 4.0753) -- (6.8076, 4.2791) -- (6.9127, 4.4948) -- (7.0178, 4.7081) -- (7.1228, 4.9174) -- (7.2279, 5.1068) -- (7.333, 5.2947) -- (7.4381, 5.4825) -- (7.5431, 5.6465) -- (7.6482, 5.8208) -- (7.7533, 6.015) -- (7.8584, 6.1774) -- (7.9634, 6.3207) -- (8.0685, 6.483) -- (8.1736, 6.6303) -- (8.2787, 6.7624) -- (8.3837, 6.8961) -- (8.4888, 7.0147) -- (8.5939, 7.1357) -- (8.699, 7.2384) -- (8.804, 7.3427) -- (8.9091, 7.4223) -- (9.0142, 7.5202) -- (9.1193, 7.6013) -- (9.2243, 7.6833) -- (9.3294, 7.7709) -- (9.4345, 7.8361) -- (9.5396, 7.9102) -- (9.6446, 7.9595) -- (9.7497, 8.0192) -- (9.8548, 8.0749) -- (9.9599, 8.1187) -- (10.0649, 8.1649) -- (10.17, 8.2102) -- (10.2751, 8.2532) -- (10.3801, 8.2946) -- (10.4852, 8.3344) -- (10.5903, 8.371) -- (10.6954, 8.4029) -- (10.8004, 8.4339) -- (10.9055, 8.4649) -- (11.0106, 8.4968) -- (11.1157, 8.527) -- (11.2207, 8.5549) -- (11.3258, 8.582) -- (11.4309, 8.6042) -- (11.536, 8.617) -- (11.641, 8.6345) -- (11.7461, 8.6448) -- (11.8512, 8.6615) -- (11.9563, 8.6751) -- (12.0613, 8.6838) -- (12.1664, 8.6982) -- (12.2715, 8.7061) -- (12.3766, 8.7165) -- (12.4816, 8.7244) -- (12.5867, 8.734) -- (12.6918, 8.7419) -- (12.7969, 8.7507) -- (12.9019, 8.761) -- (13.007, 8.7682) -- (13.1121, 8.7722) -- (13.2172, 8.7754) -- (13.3222, 8.7817) -- (13.4273, 8.7881) -- (13.5324, 8.7937) -- (13.6374, 8.7985) -- (13.7425, 8.8024) -- (13.8476, 8.8064) -- (13.9527, 8.8128) -- (14.0577, 8.8191);



      \end{scope}
    \end{scope}
  \end{scope} 
  % }

  \tikzset{shift={(-\xshift,-\yshift)}}
    

    
    \draw[<->,thick]
        (3,0) node[below left] {$t$}
        -| (0,2.5) node[left] {$N$};

    % 
    % Arrows from left to right
    \draw[->,thick] (-5,2.5) -- (-0.52,1.2);
    \draw[->,thick] (-5,.9) -- (-0.5,.9);
    \draw[->,thick] (-5,-1) -- (-0.52,.7);
    % \draw[->] (-5.5,-0.5) -- (-0.5,1);
    % \draw[->] (-5.5,-1) -- (-0.5,1);
\end{tikzpicture}
     
    \bcaption{Averaging of simulation results.}{Multiple runs from an SIR ABM are aggregated to show average infections of $N$ individuals over time $t$. Shaded intervals around final curves represent variation in individual model runs.}
    \label{fig:sir-abm-avg}
\end{figure}

It is important to note, however, that agent-based modelling is still a somewhat nascent field, and is therefore not without its growing pains and limitations. Firstly, as ABMs simulate potentially millions of agents, they are computationally more expensive compared to their analytical population-level counterparts \cite{borshchev_system_2004, axtell_agent-based_2022}. Secondly, model reproducibility has historically been a major challenge for ABMs---one problem is the lack of standardisation and guidelines for the practice \cite{collins_call_2015} despite modellers expressing the need for such standards \cite{janssen_towards_2008}. Stochasticity, while useful for modelling heterogeneity, can also contribute to issues with reproducibility in model results \cite{fitzpatrick_issues_2019}. Finally, little attention has historically been paid to evaluating the extent to which ABMs can provide valid causal insights \cite{marshall_formalizing_2015}. While efforts have been made to address these areas, for example by \citet{grimm_standard_2006} who introduced a standardised protocol for ABM description, these attempts have not been without their criticisms (e.g., see \citet{donkin_replicating_2017}). Despite these drawbacks, ABMs still provide useful alternatives to traditional models that cannot account for individual-level behaviour.

\subsubsection{Agent-based modelling for vector-borne diseases}

Due to the flexibility of the agent-based approach, there have been multiple ABMs proposed for simulating VBD spread. Some models have focused solely on vector population dynamics and disease spread \cite{jacintho_agent-based_2010, maneerat_spatial_2016, dommar_agent-based_2014, krzhizhanovskaya_agent-based_2020}, whereas others have examined public health issues, such as resistance to insecticides \cite{selvaraj_vector_2020}, re-emergence \cite{linard_multi-agent_2009}, and influence on vaccination impact projections \cite{perkins_agent-based_2019}. Spatial ABMs such as those proposed by \citet{jacintho_agent-based_2010} that model vectors as agents can incorporate important heterogeneous factors such as vector movement and behaviours (e.g., biting) at the individual level. However, the computational demand of simulating vectors as agents limits the spatial scale of such models \cite{manore_network-patch_2015, de_mooij_framework_2023}, and the lack of data around vector movement behaviours potentially makes the agent representation for vectors less sound \cite{jacintho_agent-based_2010, maneerat_spatial_2016}.

To reap the benefits of both agent- and equation-based approaches, an increasingly popular solution is to form \textit{hybrid} ABMs that couple separate models of vector and human populations. In hybrid ABMs, multiple agent- and equation-based models (called \textit{submodels}) are coupled through some mechanism in which output from one submodel is input to another \cite{borshchev_system_2004}. Studies have suggested hybrid epidemiological models can dramatically save computational resources without losing a significant amount of fidelity when population sizes are sufficiently large \cite{bobashev_hybrid_2007, hunter_comparison_2018, hunter_hybrid_2020}. In VBD contexts, the agent-based component usually consists of humans, and the equation-based submodels draw on years of mathematical VBD modelling to represent vector populations \cite{mniszewski_towards_2014, manore_network-patch_2015, mateus_c_modeling_2021}.

A novel architecture for a hybrid VBD ABM was proposed by \citet{manore_network-patch_2015} and \citet{mniszewski_towards_2014}, who introduced a general network-patch methodology to incorporate spatial heterogeneity and host movement in VBD models. The authors extended work from \citet{adams_man_2009} to represent important factors of variation in mosquito density within the equation-based submodel, and built on ideas from \citet{perkins_heterogeneity_2013} to determine mobility of agents across the proposed network structure. In all of the aforementioned ABMs, human (or agent) behaviour is implicitly encoded: for example, the model proposed in \citet{manore_network-patch_2015} assumes people move between locations (e.g., home residence, school) at random, and \citet{dommar_agent-based_2014} assume people are either born \q{travellers} or \q{non-travellers} where non-travellers remain in one location for the entire simulation. These underlying assumptions of behaviour are similar to the well-mixed assumption of ODE models in that they are clearly false, but useful for simplifying complex systems.

\subsubsection{Modelling human behaviour}

Representing behavioural decision-making processes in ABMs is an understudied area in the modelling domain. For example, \citet{tully_coevolution_2013} incorporated risk perception within agents for their model of HIV spread, but the authors conceded that further work should \q{build greater realism into the decision-making process.} Similarly, \citet{du_how_2021} explicitly modelled agents' perceived susceptibility to infection in their model for influenza spread, but used a global threshold to determine whether agents would adopt preventive measures or not. A similar threshold method of modelling preventive behaviours was used in \citet{mao_modeling_2014}, although the threshold value was varied over agents. The majority of models that simulate preventive behaviours based on agent-level risk perception, such as those above, use traditional opinion dynamics models in which agents update their beliefs through interactions with other agents or exposure to external information \cite{du_how_2021, mao_modeling_2014, brainard_agent-based_2020, yu_how_2024}.

In recent years, however, more attention has been paid to embedding human behaviour in agents to simulate realistic decision-making within ABMs. \citet{scheidegger_agent-based_2017}, for example, incorporated human behaviour in a VBD model to show how small changes in behaviour can have large impacts on disease propagation. \citet{barbrook-johnson_uses_2017} simulated agents in an influenza epidemic and designed a threshold-based decision-making process for agents to adopt preventive behaviours designed around PMT and the HBM. Similarly, \citet{abdulkareem_intelligent_2018} extended an ABM of cholera transmission by augmenting agents with decision-making processes based on PMT to determine preventive measure adoption. The same ABM was also used in \citet{abdulkareem_risk_2020} to explore how social structures of learning influenced preventive measure adoption and disease spread. These examples exemplify what \citet{de_mooij_framework_2023} call \textit{deliberative agents}---agents that are capable of engaging in autonomous decision-making. However, such ABMs that encode psychologically-grounded behavioural theories are still rare, and the implications of modelling behaviour in different ways has not yet been extensively investigated.

%and there is evidently a need to quantify the implications of different ways of modelling behaviour.
